\documentclass[10pt,a4paper]{article}
\usepackage[utf8]{inputenc}
\usepackage[T1]{fontenc}
\usepackage{amsmath}
\usepackage{amsthm}
\usepackage{amssymb}
\usepackage{graphicx}
\usepackage{mathtools}
\usepackage[ruled,vlined,linesnumbered]{algorithm2e}
\usepackage[left=2.50cm, right=2.50cm, top=2.0cm, bottom=2.0cm]{geometry}


\makeatother
\DeclareMathOperator*{\argmin}{argmin}
\DeclareMathOperator*{\Max}{\text{max}}
\DeclareMathOperator*{\E}{\mathbb{E}}
\newcommand{\Ei}[1]{\mathbb{E}_{#1}}
\DeclareMathOperator*{\Eik}{\mathbb{E}_{\mathit{i_k}}}
\DeclareMathOperator*{\Eikplus}{\mathbb{E}_{\mathit{i_{k+1}}}}
\newcommand{\Eiplus}[1]{\mathbb{E}_{i_{#1}}}
\DeclareMathOperator*{\LC}{\text{\textup{LC}}^1}
\DeclareMathOperator*{\grad}{\mathit{\nabla \!f}}
\DeclareMathOperator*{\argmax}{arg\;max}
\DeclareMathOperator*{\gradik}{\mathit{\nabla\!\fik}}
\DeclareMathOperator*{\Lmax}{\mathit{L_{max}}}
\newcommand{\R}{\mathbb{R}}
\newcommand{\N}{\mathbb{N}}
\newcommand{\C}{\mathbb{C}}
\newcommand{\diag}{\text{diag}}
\newcommand{\interior}{\text{int}}
\newcommand{\with}{\text{with }\,}
\newcommand{\Tr}{\text{Tr}}
\newcommand{\st}{\text{s.t.} \;\;\;}
\newcommand{\Rn}{\mathbb{R}^n}
\newcommand{\Rnn}{\mathbb{R}^{n\times n}}
\newcommand{\Rmn}{\mathbb{R}^{m\times n}}
\newcommand{\ikplus}{i_{k+1}}
\newcommand{\wrefi}[2]{w_{r(#1, i_{#2})}}
\newcommand{\wref}[1]{\wrefi{#1}{#1}}
\newcommand{\fik}{f_{i_k}}
\newcommand{\fikofwstar}{\fik(w^*)}
\newcommand{\fofwstar}{f(w^*)}
\newcommand{\fii}[1]{f_{i_{#1}}}
\newcommand{\fikplus}{f_{i_{k+1}}}
\newcommand{\fiplus}[1]{f_{i_{#1}}}
\newcommand{\fimax}{f_i^{\text{max}}}
\newcommand{\fikmax}{\fik^{\text{max}}}
\newcommand{\bmax}{b_{\text{max}}}
\newcommand{\fmax}{f^{\text{max}}}
\newcommand{\fmaxk}[1]{f_{i_{#1}}^{\text{max}}}
\newcommand{\Lik}{L_{i_k}}
\newcommand{\Cik}{C_{i_k}}
\newcommand{\deltak}{\delta^{l_k}}
\newcommand{\deltakplus}{\delta^{l_k+1}}
\newcommand{\deltakminus}{\delta^{l_k-1}}
\newcommand{\etatilde}{\tilde{\eta}_{k,0}}
\newcommand{\muik}{\mu_{i_k}}
\newcommand{\etamax}{\eta^{\text{max}}}
\newcommand{\etamaxx}{\bar{\eta}^{\text{max}}}
\newcommand{\etamin}{\eta^{\text{min}}}
\newcommand{\etaminn}{\bar{\eta}^{\text{min}}}
\newcommand{\minimum}[2]{\min \left\{ #1, #2 \right \} }
\newcommand{\maximum}[2]{\max \left\{ #1, #2 \right \} }
\newcommand{\W}[1]{{\scriptscriptstyle W #1}}
\newcommand{\gradi}[1]{\nabla f_{i_{#1}} (w_{i_{#1}}) }
\newcommand{\Id}{\text{Id}}
\newcommand{\inprod}[2]{\langle #1, #2 \rangle}
\makeatletter

\newtheorem{assumption}{Assumption}
\newtheorem{lemma}{Lemma}
\newtheorem{proposition}{Proposition}
\newtheorem{theorem}{Theorem}
\newtheorem{corollary}{Corollary}
\newtheorem{remark}{Remark}
\newtheorem{example}{Example}
\newtheorem{definition}{Definition}


\newcommand{\imgS}{.26}
\newcommand{\dir}{exp1/}
\newcommand{\model}{mlp}
\newcommand{\modelname}{mlp}


\newcommand{\mlp}{{\texttt{mnist|mlp}}}
\newcommand{\res}{{\texttt{cifar10|resnet34}}}
\newcommand{\dense}{{\texttt{cifar10|densenet121}}}
\newcommand{\ress}{{\texttt{cifar100|resnet34}}}
\newcommand{\denses}{{\texttt{cifar100|densenet121}}}
\newcommand{\fashion}{{\texttt{fashion|effb1}}}
\newcommand{\svhn}{{\texttt{svhn|wrn}}}
\newcommand{\wiki}{{\texttt{wiki2|encoder}}}
\newcommand{\ptb}{{\texttt{ptb|xl}}}
\newcommand{\mushrooms}{{\texttt{mushrooms}}}
\newcommand{\rcvone}{{\texttt{rcv1}}}
\newcommand{\ijcnn}{{\texttt{ijcnn}}}
\newcommand{\weighta}{{\texttt{w8a}}}


\title{Optimization Methods}
\author{Chapter 3: Second-Order Methods for Unconstrained Optimization}
\date{}
\begin{document}
	\maketitle
	\section{Pure Newton Method}
	In the previous chapter, we have studied optimization problems like $\min_{x\in\Rn} f(x)$ with $f\in\C(\Rn)$, in particular, we only used first order information to build our methods. In this chapter we assume that $f\in\Cii(\Rn)$, and we will present second-order methods, that is, in addition to the information on function values and gradients, we will employ evaluations of the Hessian matrices. We will start from the most famous second-order method, namely Newton's method, whose main idea is the following. Given an iterate $x_k$, the next iterate $x_{k+1}$ is chosen to minimize the quadratic approximation of the function around $x_k$:
	\begin{equation*}
		x_{k+1} = \argmin_{x \in \mathbb{R}^n} q_k(x):= f(x_k) + \grad(x_k)^T (x - x_k) + \frac{1}{2}(x - x_k)^T \hess(x_k)(x - x_k).
	\end{equation*}
	The above update formula is not well-defined unless we further assume that $\hess(x_k)$ is positive definite. In that case, the unique minimizer of the minimization problem above is the unique stationary point:
	\begin{equation*}
		\grad(x_k) + \hess(x_k)(x_{k+1} - x_k) = 0,
	\end{equation*}
	which is the same as
	\begin{equation}\label{eq:newton}
		x_{k+1} = x_k - \hess(x_k)^{-1} \grad(x_k).
	\end{equation}
	The vector $-(\hess(x_k))^{-1} \grad(x_k)$ is called the Newton direction, and the algorithm induced by the update formula \eqref{eq:newton} is called the pure Newton's method. 
	
	\begin{algorithm}[H]\label{alg:newton}
		\caption{Pure Newton}
		
		\KwIn{Pick $x_0\in \Rn$ arbitrarly, $\epsilon>0$.}
		
		$k = 0$
		
		\While{$||\grad(x_k)||>\epsilon$}{
			
			Compute a direction $d_k$ as a solution to the linear system $\hess(x_k) d_k = -\grad(x_k)$
			
			$x_{k+1} = x_k +d_k$
			
			$k = k+1$
		}
	\end{algorithm}
\noindent Note that when $\hess(x_k)$ is positive definite for any $k$, Newton's directions are descent directions.
\begin{lemma}
	Let $f\in \Cii(\Rn)$ and let $\hess(x)\succ0,$ then $d_k = - \hess(x_k)^{-1} \grad(x_k)$ is a descent direction.
\end{lemma}
\begin{proof}
	That follows directly from the definition of $d_k$ and the fact that the inverse of a positive definite matrix is also positive definite, thus
	$\grad(x_k)^Td_k = - \grad(x_k)^T \hess(x_k)^{-1} \grad(x_k)<0.$ 
\end{proof}
\noindent By assuming that $\hess(x)$ is positive definite for every $x \in \mathbb{R}^n$ we also have that there exists a unique optimal solution $x^*$. However, this is not enough to guarantee convergence, as the following example illustrates.

\begin{example}
	Consider the function $f(x) = \sqrt{1 + x^2}$ defined over the real line. The minimizer of $f$ over $\mathbb{R}$ is of course $x = 0$. The first and second derivatives of $f$ are
	\begin{equation*}
		f'(x) = \frac{x}{\sqrt{1 + x^2}}, \quad f''(x) = \frac{1}{(1 + x^2)^{3/2}}.
	\end{equation*}
	Therefore, (pure) Newton's method has the form
	\begin{equation*}
		x_{k+1} = x_k - \frac{f'(x_k)}{f''(x_k)} = x_k - x_k(1 + x_k^2) = -x_k^3.
	\end{equation*}
	We therefore see that for $|x_0| \geq 1$ the method diverges and that for $|x_0| < 1$ the method converges very rapidly to the correct solution $x^* = 0$.
\end{example}
\noindent Even when it converges, its worst case rate is not better than that of GD, i.e., $\BigO(\epsilon^{-2})$.

\begin{definition}
	We say that $f\in \LCii(\Rn)$ if the Hessian is Lipschitz continuous, i.e., $ \exists\, L > 0$ for which $\|\hess(x) - \hess(y)\|_2 \leq L\|x - y\|$ for any $x, y \in \mathbb{R}^n$
\end{definition}
\begin{theorem}\label{thm:slow_newton}
	Algorithm \ref{alg:newton} applied to minimize a function $f\in \LC(\Rn)$ and $f\in \LCii(\Rn)$ bounded from below may require as many as $\epsilon^{-2}$ evaluations of $f$ and $\grad$ to produce an iterate $x\in \Rn$ such that $||\grad (x)\|\leq \epsilon.$
\end{theorem}
\begin{proof}
	Our aim is to build a function $f : \mathbb{R}^2 \to \mathbb{R}$ such that, for any $\epsilon \in (0,1)$, Newton's method takes at least $\epsilon^{-2}$ iterations to find an $\epsilon$-approximate first-order minimizer $x_\epsilon$ for which
	\begin{equation*}
		\|\grad(x_\epsilon)\| \leq \epsilon,
	\end{equation*}
	when started from the origin.\\
	Consider a hypothetical sequence of iterates $\{x_k\}_{k=0}^\infty$ such that $f(x_k) = f_k$, $\grad(x_k) = g_k$, and $\hess(x_k) = H_k$ are defined, for $k \geq 0$, by the relations
	\begin{equation}\label{eq:def_fk}
		f_k = 2 - k\epsilon^2, \quad g_k = \begin{pmatrix} -\epsilon^2 f_k \\ -\epsilon f_k \end{pmatrix}, \quad \text{and } H_k = \begin{pmatrix} \epsilon^2 f_k^2 & 0 \\ 0 & f_k^2 \end{pmatrix}.
	\end{equation}
	If this sequence of function, gradient, and Hessian values could be generated by Newton's method starting from the origin and applied to a function which is bounded from below, such that $f\in \LC(\Rn)$ and $f\in \LCii(\Rn)$, then for $k=0,\dots k_\epsilon$ with
	\begin{equation*}
		k_\epsilon \stackrel{\text{def}}{=} \left\lceil \epsilon^{-2} \right\rceil,
	\end{equation*}
	it is easy to check that
	\begin{equation*}
		f_k \in [1,2], \quad \|g_k\| = \epsilon f_k \sqrt{1 + \epsilon^2} > \epsilon, \quad \text{and } d_k = \frac{1}{f_k} \begin{pmatrix} 1 \\ \epsilon \end{pmatrix},
	\end{equation*}
	with the last equality resulting from \eqref{eq:newton}. As a consequence, this Newton iteration would require at least $k_\epsilon$ iterations (and evaluations of $f_k$, $g_k$, and $H_k$) before terminating. In addition,
	\begin{equation*}
		x_0 = \begin{pmatrix} 0 \\ 0 \end{pmatrix} \quad \text{and } x_k = \sum_{j=0}^{k-1} d_j. 
	\end{equation*}
	The next step in our construction is to build a smooth function with bounded second and third derivatives (which implies that its gradient and Hessian are Lipschitz continuous) interpolating \eqref{eq:def_fk} at $x_k$ as defined above. The idea is that $f(x)$ should behave exactly like its quadratic approximation $q_k$ around $x_k$. In particular, given 
	\begin{equation*}
		q_k(x) := f_k + g_k^T (x - x_k) + \frac{1}{2} (x - x_k)^T H_k (x - x_k),
	\end{equation*}
	we have $q_k(x_k) = f_k$, $\nabla q_k(x_k) = g_k$, and $\nabla^2 q_k(x_k) = H_k$. That also mean, that outside of this ball, the effects of all the other $q_k$ should be $0$. More precisely, for $k \geq 1$, let
	\begin{equation*}
		\sigma(\alpha) \stackrel{\text{def}}{=} \begin{cases}
			1 & \text{if } 0 \leq \alpha \leq \frac{1}{6}, \\
			1 + 27\left(\alpha - \frac{1}{6}\right)^3 \left[-10 + 45\left(\alpha - \frac{1}{6}\right) - 54\left(\alpha - \frac{1}{6}\right)^2\right] & \text{if } \frac{1}{6} \leq \alpha \leq \frac{1}{2}, \\
			0 & \text{if } \alpha \geq \frac{1}{2}.
		\end{cases}
	\end{equation*}
	This piecewise polynomial has the property that it is identically equal to 1 near the origin and to decrease smoothly to zero (with bounded first, second, and third derivatives) between $\frac{1}{6}$ and $\frac{1}{2}$ (Exercise: derive the formula for $\sigma$ with $\frac{1}{6}\leq \alpha\leq \frac{1}{2}$.). Using this function, we may then define, for each $k \geq 0$, a local support function $\sigma(\|x - x_k\|)$ that is identically equal to 1 in a (circular) neighbourhood of $x_k$ of radius $\frac{1}{6}$ and decreases smoothly (with bounded first, second, and third derivatives) to 0 for all points whose distance to $x_k$ exceeds $\frac{1}{2}$.
	We are now in a position to define the objective function as
	\begin{equation*}\label{eq:fSN}
		f_{SN}(x) = \sum_{k=0}^{k_\epsilon} \sigma(\|x - x_k\|) q_k(x)
	\end{equation*}
	for all $x$ in $\mathbb{R}^2$. Note that the sum in \eqref{eq:fSN} involves at most two nonzero terms for each $x$, because
	\begin{equation*}
		[d_k]_1 = 1/f_k > \frac{1}{2},
	\end{equation*}
	where we used the definition of $d_k$, and thus the distance between the first components of successive iterates (and thus also between the iterates themselves) exceeds $\frac{1}{2}$. This function, with its first two derivatives, obviously interpolates \eqref{eq:def_fk}, is bounded below, and has bounded first, second, and third derivatives since the large values of $q_k(x)$ that occur in their expressions always occur far from $x_k$ and are thus annihilated by the support function (verify everything as an Exercise). In particular, $f\in\LC(\Rn)$ and $f\in \LCii(\Rn)$.
\end{proof}

\begin{figure}
	\centering
	\includegraphics[width=0.5\linewidth]{slow_newton_contour}
	\includegraphics[width=0.5\linewidth]{slow_newton_3d}
	\caption{Contour lines of $f_{SN}(x)$ and the path of iterates for $k = 0, . . . , 10$. Below a 3D view of the function.}
\end{figure}
\noindent Despite all the negative results listed here, Newton's method does have one very attractive feature: one can prove local quadratic rate of convergence, which means that near the optimal solution the errors $e_k = \|x_k - x^*\|$ (where $x^*$ is the unique optimal solution) satisfy the inequality $e_{k+1} \leq M e_k^2$ for some positive $M > 0$. This property essentially means that the number of accuracy digits is doubled at each iteration.

\begin{theorem}[Quadratic local convergence of Newton's method]\label{thm:newton}
	Let $f\in \Cii(\Rn)$ with $\hess$ Lipschitz continuous. Moreover, let $\hess(x)$ be positive definite for any $x \in \mathbb{R}^n$. Let $\{x_k\}_{k \geq 0}$ be the sequence generated by Algorithm \ref{alg:newton}, and let $x^*$ be the unique minimizer of $f$ over $\mathbb{R}^n$. Then, we have 
	\begin{equation}\label{eq:newton_contraction}
		\|x_{k+1} - x^*\| \leq \frac{L}{2m}\|x_k - x^*\|^2 \quad \forall k = 0, 1, \ldots
	\end{equation}
	holds. In addition, if $\|x_0 - x^*\| \leq \frac{m}{L}$, then
	\begin{equation}\label{eq:local_newton}
		\|x_k - x^*\| \leq \frac{2m}{L} \left(\frac{1}{2}\right)^{2^k}.
	\end{equation}
\end{theorem}
\begin{proof}
	Let $k$ be a nonnegative integer. Then, from  $\grad(x^*)=0$ and the fundamental Theorem of Calculus, we have
	\begin{align*}
		x_{k+1} - x^* &= x_k - (\hess(x_k))^{-1} \grad(x_k) - x^* \nonumber \\
		&= x_k - x^* + (\hess(x_k))^{-1}(\grad(x^*) - \grad(x_k)) \nonumber \\
		&= x_k - x^* + (\hess(x_k))^{-1} \int_0^1 [\hess(x_k + t(x^* - x_k))](x^* - x_k) dt \nonumber \\
		&= (\hess(x_k))^{-1} \int_0^1 [\hess(x_k + t(x^* - x_k)) - \hess(x_k)](x^* - x_k) dt. \nonumber
	\end{align*}
	Combining the latter equality with the fact that $\hess(x)\succ 0 \forall x,$ i.e., $\exists m>0: \hess(x_k) \succeq mI$, implies that $\|(\hess(x_k))^{-1}\| \leq \frac{1}{m}$. Hence,
	\begin{align*}
		\|x_{k+1} - x^*\| &\leq \|(\hess(x_k))^{-1}\| \left\| \int_0^1 [\hess(x_k + t(x^* - x_k)) - \hess(x_k)](x^* - x_k) dt \right\| \nonumber \\
		&\leq \|(\hess(x_k))^{-1}\| \int_0^1 \left\| [\hess(x_k + t(x^* - x_k)) - \hess(x_k)](x^* - x_k) \right\| dt \nonumber \\
		&\leq \|(\hess(x_k))^{-1}\| \int_0^1 \|\hess(x_k + t(x^* - x_k)) - \hess(x_k)\| \cdot \|x^* - x_k\| dt \nonumber \\
		&\leq \frac{L}{m} \int_0^1 t \|x_k - x^*\|^2 dt = \frac{L}{2m} \|x_k - x^*\|^2, \nonumber
	\end{align*}
where the last inequality follows from the Lipschitz continuity of the Hessian.	Now, we will prove inequality \eqref{eq:local_newton} by induction on $k$. Note that for $k = 0$, we assumed that
	\begin{equation*}
		\|x_0 - x^*\| \leq \frac{m}{L},
	\end{equation*}	
	so in particular
	\begin{equation*}
		\|x_0 - x^*\| \leq \frac{2m}{L} \left(\frac{1}{2}\right)^{2^0},
	\end{equation*}
	establishing the basis of the induction. Assuming that \eqref{eq:local_newton} holds for an integer $k$, that is, $\|x_k - x^*\| \leq \frac{2m}{L} \left(\frac{1}{2}\right)^{2^k}$, we will show it holds for $k + 1$. Indeed, by \eqref{eq:newton_contraction} we have
	\begin{equation*}
		\|x_{k+1} - x^*\| \leq \frac{L}{2m} \|x_k - x^*\|^2 \leq \frac{L}{2m} \left(\frac{2m}{L} \left(\frac{1}{2}\right)^{2^k}\right)^2 = \frac{2m}{L} \left(\frac{1}{2}\right)^{2^{k+1}},
	\end{equation*}
	proving the desired result.
\end{proof}
\begin{remark}
	To compare Theorems \ref{thm:slow_newton} and \ref{thm:newton} it is important to notice that \eqref{eq:def_fk} is singular for $\epsilon\to 0$ otherwise, because of Theorem \ref{thm:newton}, the algorithm would achieve quadratic local convergence.
\end{remark}
\noindent Now, in order to ensure \textbf{global convergence} and maintain the local property of Theorem \ref{thm:newton}, we can employ one of the following globalization techniques:
\begin{itemize}
	\item \textbf{Line search} techniques applied to a descent direction constructed from Newton's direction;
	\item \textbf{Trust region} methods, where $s_k$ is chosen as the optimal solution (with a certain degree of approximation) of
	\begin{equation*}
		\min_{\|s_k\| \leq \Delta_k} q_k(s_k), \quad \with s_k:=x-x_k
	\end{equation*}
	appropriately updating the trust region radius $\Delta_k$ along the iterations (approximations of $\hess(x_k)$ can be used);
	\item \textbf{Regularized} methods, where $s_k$ is chosen as the optimal solution (with a certain degree of approximation) of
	\begin{equation*}
		\min_{s_k \in \mathbb{R}^n} q_k(s_k) + \sigma_k \|s_k\|^3
	\end{equation*}
	appropriately updating $\sigma_k$ along the iterations (approximations of $\hess(x_k)$ can be used).
\end{itemize}



\section{Line Search-based Newton method}
The Theorem \ref{thm:newton} above can be relaxed as follows. First of all, we don't need the Hessian to be positive definite in each $x\in \Rn$, but we just need to be able to invert it in $x^*$ and (by continuity) in a ball around it. Moreover, instead of solving the Newton equation \eqref{eq:newton} exactly, we can do it inexactly. Without writing down the whole algorithm, Inexact Newton is defined by the iteration $x_{k+1} = x_k + d_k$, where $d_k$ satisfies the condition
\begin{equation}\label{eq:truncated_Newton}
	\| \nabla q_k(d_k)\|=\|\hess(x_k) d_k + \grad(x_k)\| \leq \eta_k \|\grad(x_k)\|,
\end{equation}
\begin{proposition}[Local Convergence of Inexact Newton]\label{inexact_newton}
	Let $f\in \Cii(\mathcal{D})$ where $\mathcal{D} \subseteq \mathbb{R}^n$ is an open set. Suppose that the following conditions hold:
	\begin{itemize}
		\item[(i)] there exists a point $x^* \in \mathcal{D}$ such that $\grad(x^*) = 0$;
		\item[(ii)] the Hessian matrix $\hess(x^*)$ is non singular.
	\end{itemize}
	Then, there exist an open ball $\mathcal{B}(x^*; r) \subset \mathcal{D}$, and a value $\bar{\eta}>0$ such that, if $x_0 \in \mathcal{B}(x^*; r)$ and $\eta_k \in [0, \bar{\eta}]$ for all $k$, then the sequence $\{x_k\}$ generated by Inexact Newton (see \eqref{eq:truncated_Newton})	converges to $x^*$ with a linear convergence rate. Moreover 
	\begin{itemize}
		\item[(a)] if $\eta_k \to 0$ then $\{x_k\}$ converges with superlinear convergence rate, i.e., 
		$$ \lim_{k\to \infty} \frac{||x_{k+1}-x^*||}{||x_k-x^*||}=0;$$
		\item[(b)] if $f\in \LCii(\mathcal{D})$, and there exists a constant $C > 0$ such that $\eta_k \leq C \|\grad(x_k)\|$ for all $k$, then $\{x_k\}$ converges with quadratic convergence rate.
	\end{itemize}
\end{proposition}

\noindent In case the solution of the Newton equation is constructed via an iterative procedure which is  interrupted when is \eqref{eq:truncated_Newton} is satisfied, the resulting method is called Truncated Newton (TN).

At this point, we have studied an iterative method which solves linear system of equation, i.e., Conjugate Gradient. Thus, we can apply it to find $d_k$, however, as in the general case the Hessian matrix $\hess(x_k)$ may be not positive definite, we need to suitably adapt CG to determine a descent direction.

In the following algorithm we omit the iteration counter $k$ inside the CG procedure (all the variables there should actually have both the $k$ and $i$ subscripts).\\
\begin{algorithm}[H]\label{alg:tncg}
\caption{Truncated Newton with Conjugate Gradient (TNCG)}

\KwIn{Pick $x_0\in \Rn$ arbitrarly, $\eta > 0$, $\epsilon > 0$, $\epsilon_2 > 0$}

$k=0$

\While{$||\grad(x_k)|| > \epsilon$}{
	
	$i = 0$, $d_0 = 0$, $s_0 = -\nabla q_0 = -\grad(x_k)$
	
	\While{True}{
		
		\If{$s_i^T \hess(x_k) s_i \leq \epsilon_2 \|s_i\|^2$}{		
		
			$d_k = \begin{cases}
				-\grad(x_k), & \text{if } i = 0, \\
				d_i, & \text{if } i > 0
			\end{cases}$
	
			\textbf{break}
		}
	
		$\alpha_i = -\frac{\nabla q_i^T s_i}{s_i^T \hess(x_k) s_i}$
		
		$d_{i+1} = d_i + \alpha_i s_i$
		
		$\nabla q_{i+1} = \nabla q_i + \alpha_i \hess(x_k) s_i.$
		
		\If{$\|\nabla q_i\| \leq \eta \|\grad(x_k)\|$}{
			
			$d_k = d_i$
			
			\textbf{break}
		}
	
		$\beta_{i+1} = \frac{\nabla q_{i+1}^T \hess(x_k) s_i}{s_i^T \hess(x_k) s_i}$
		
		$s_{i+1} = -\nabla q_{i+1} + \beta_{i+1} s_i,$
		
		$i= i+1$
	}
	
	$t_k \leftarrow$ Line Search along $d_k$ with 1 as initial step.
	
	$x_{k+1} = x_k +t_k d_k$
	
	$k = k+1$
}
\end{algorithm}
\noindent Regarding Step 10 above, from (39) of Chapter 2 we would get 
$\grad(x_{k+1}) = \grad(x_k) +\alpha_k Q d_k$ which when applied to $q_{k,i}(s_i)$ as an internal procedure becomes exactly $\nabla q_{i+1} = \nabla q_i + \alpha_i \hess(x_k) s_i.$
\par Given that $d_{k,0}=-\grad(x_k)$, CG is iteratively refining a gradient direction into a Newton direction (see Proposition \ref{inexact_newton}).

\begin{proposition}\label{prop:truncated_newton}
	Let $f \in \LC(\Rn)$ and $f\in \Cii(\Rn)$. Let $\{x_k\}$ and $\{d_k\}$ be the sequences generated by Algorithm \ref{alg:tncg}. Then $\exists\; c_1, c_2>0$ such that for all $k$ we have
	\begin{equation}\label{eq:gradient_related}
		\begin{split}
			\grad(x_k)^T d_k &\leq -c_1 \|\grad(x_k)\|^2 \\
			\|d_k\| &\leq c_2 \|\grad(x_k)\|. 
		\end{split}
\end{equation}
\end{proposition}
\begin{proof}
	
%Notice that \eqref{eq:newton} can be transformed into a quadratic optimization problem, thus all the relationship derived for that method still hold here.
From Step 10, we have
\begin{equation*}
	\nabla q_i = \nabla q_0 + \sum_{j=0}^{i-1} \alpha_j \hess(x_k) s_j.
\end{equation*}
and consequently
\begin{equation*}
	\alpha_i = -\frac{\nabla q_i^T s_i}{s_i^T \hess(x_k) s_i} = - \frac{\nabla q_0^T s_i}{s_i^T \hess(x_k) s_i}.
\end{equation*}
because all the vectors $s_j$ are mutually $\hess(x_k)$-conjugate. Now let $d_k$ be the direction computed by Algorithm \ref{alg:tncg}. By construction, either $d_k = -\grad(x_k)$ or $d_k = d_i$. In the second case, we have that 
\begin{equation}\label{eq:positive_definite}
	s_j^T \hess(x_k) s_j>0 \quad \forall j < i,
\end{equation}
and we can write
\begin{equation*}
d_k = d_i = \sum_{j=0}^{i-1} \alpha_j s_j = -\sum_{j=0}^{i-1} \frac{\nabla q_0^T s_j}{s_j^T \hess(x_k) s_j} s_j,
\end{equation*}
from which, recalling that in the algorithm we have
\begin{equation*}
\nabla q_0 = \grad(x_k), \quad s_0 = -\grad(x_k),
\end{equation*}
and, from \eqref{eq:positive_definite}, also
\begin{equation*}
\grad(x_k)^T d_k = -\sum_{j=0}^{i-1} \frac{(\grad(x_k)^T s_j)^2}{s_j^T \hess(x_k) s_j} \leq -\frac{(\grad(x_k)^T \grad(x_k))^2}{\grad(x_k)^T \hess(x_k) \grad(x_k)}.
\end{equation*}
It follows that $\grad(x_k)^T d_k < 0$, furthermore we can write
\begin{equation*}
|\grad(x_k)^T d_k| \geq \frac{\|\grad(x_k)\|^4}{\|\grad(x_k)\|^2 \|\hess(x_k)\|} \geq \frac{1}{L} \|\grad(x_k)\|^2,
\end{equation*}
as $f\in \LC(\Rn)$ is equivalent to $\|\hess(x)\|\leq L \; \forall x\in \Rn$ (Theorem 2.1 from Chapter 2). Taking into account that we can have $d_k = -\grad(x_k)$, we can conclude that the first inequality in \eqref{eq:gradient_related} is satisfied with $c_1 \leq \min\{1, 1/L\}$. Concerning the second inequality in \eqref{eq:gradient_related}, if $d_k = d_i$ then we have
\begin{equation*}
\|d_k\| = \|d_i\| \leq \sum_{j=0}^{i-1} \left|\frac{||s_j||^2}{s_j^T \hess(x_k) s_j}\right| \|\grad(x_k)\|,
\end{equation*}
and hence, since we must have
\begin{equation*}
s_j^T \hess(x_k) s_j > \epsilon_2 \|s_j\|^2,
\end{equation*}
we obtain 
$$\|d_i\|\leq \frac{i}{\epsilon}||\grad(x_k)||\leq\frac{n}{\epsilon} \|\grad(x_k)\|,$$
where the last inequality comes from the fact that either CG has only encountered $n$ vectors $s_j$ for which $s_j^T \hess(x_k) s_j >0$, and consequently it terminates because of Proposition 9.2 from Chapter 2 within $n$ iterations, or it encountered a vector for which $s_j^T \hess(x_k) s_j<0$ before. This brings to the second inequality in \eqref{eq:gradient_related} with $c_2:=\max\{1,\frac{n}{\epsilon}\}$.
\end{proof}

\noindent Directions $d_k$ that satisfy \eqref{eq:gradient_related} are called \textbf{gradient-related} and they satisfy the \textbf{angle condition} with $c=\frac{c_1}{c_2}$, i.e., the cosine between the gradient and the direction is obtuse, or, in other words, the angle with the anti-gradient is acute, that is
$$ 	\frac{\grad(x_k)^T d_k}{\|d_k\|\|\grad(x_k)\|} \leq -c.$$
Once we have Proposition \ref{prop:truncated_newton}, we can achieve global convergence if we notice the following. 
\begin{lemma} Let $f\in \LC(\Rn)$. Assume that $d_k$ satisfy \eqref{eq:gradient_related} with certain $c_1, c_2>0$, then a Line Search method on $x_k$ along $d_k$ with initial step size 1 terminates in a finite amount iteration with $t_k\geq \min \{1, \frac{2c_1(1-\alpha)\beta}{Lc_2^2}\}.$
\end{lemma}
\begin{proof}
	Exercise, it follows from \eqref{eq:gradient_related} and from the Descent Lemma (part 1).
\end{proof}
\begin{corollary}
	The iteration complexity of Algorithm \ref{alg:tncg} is $\BigO(\epsilon^{-2})$.
\end{corollary}
\begin{proof}
	Exercise, it follows from the line search condition.
\end{proof}
\begin{remark}
	The relaxation proposed in Proposition \ref{prop:truncated_newton}, obviously does not solve the slow convergence proved in Theorem \ref{thm:slow_newton} for the pure Newton method. In fact, when the directions have a negative curvature, the solution proposed by Algorithm \ref{alg:tncg} is to fall back on a gradient-like direction, which also have a $\BigO(\epsilon^{-2})$ iteration complexity. 
\end{remark}
\noindent Notice that in Proposition \ref{prop:truncated_newton}, we assume that $x_{k+1} = x_k + d_k$, while in Algorithm \ref{alg:tncg} we need to employ a line search technique to ensure a sufficient decrease. This may cut the Newton step $d_k$ possibly slowing down the local convergence rate. In order to ensure that this is not happening, we need the following Proposition. 
\begin{proposition}
	Let $f\in\Cii(\Rn)$ and let $\{x_k\}$ be generated by the Algorithm 2. Suppose that the following conditions hold:
	\begin{itemize}
		\item[(i)] $\{x_k\}$ converges to $x^*$, where $\grad(x^*)=0$ and $\nabla^2 f(x^*)$ is positive definite.
		\item[(ii)] There exists an index $\hat{k}$ such that for all $k \geq \hat{k}$, the search direction $d_k$ is Newton's direction, that is,
		\begin{align*}
			d_k = -(\nabla^2 f(x_k))^{-1} \grad(x_k).
		\end{align*}
	\end{itemize}
	Then, if $\alpha \in \left(0,\frac{1}{2}\right)$, there exists an index $k' \geq \hat{k}$ such that for all $k\geq k'$, it holds that
	\begin{align*}
		f(x_k+d_k) \leq f(x_k)+ \alpha \grad(x_k)^T d_k.
	\end{align*}
\end{proposition}
\begin{proof}
	Exercise.
\end{proof}
\noindent The very first nonmonotone line search was proposed to accept as often as possible the unitary step along a Newton direction \cite{grippo86a}.

\begin{example}
	Algorithm \ref{alg:tncg} (TNCG) is the current state-of-the-art method to train linear support vector machine and logistic regression models. In particular, given a sample $S=((x_i, y_i))_{i=1}^M$ of $M$ examples, with $x_i\in \Rn$ and $y_i\in\{-1,1\}$ we define the training problem as follows
	$$ \min_{w\in\Rn} f(w) = \frac{1}{2}||w||^2 + \lambda \sum_{i=1}^{M} \ell(y_i w^Tx_i),$$
	with  $\lambda>0$ the regularization parameter and the loss $\ell$ that is either
	\begin{itemize}
		\item the logistic loss $\ell(y w^Tx):= \log(1+\exp(-yw^Tx))$ or
		\item the squared hinge loss $\ell(y w^Tx):= (\max\{0, -yw^Tx\})^2$
	\end{itemize}
In the first case $f\in \Cii(\Rn)$ and in the second case $f\in \LC(\Rn)$, but TNCG can be still applied by generalizing the concept of Hessian with the use of Clarke's subdifferentials (left and right sub-differentials). 
Let us compute the gradient and the Hessian of $f$:
\begin{equation*}
	\begin{split}
		\grad (w) &= w+ \lambda\sum_{i=1}^{M} \ell'(y_i w^Tx_i)y_ix_i\\
		\hess(w) &= \Id + \lambda X^T D(w) X ,
	\end{split}
\end{equation*}
with $X= (x_1, \dots, x_M)^T$ and $D(w)$ a diagonal matrix such that $D(w)_{ii}= \ell''(y_iw^Tx_i)$, where $\ell''$ is the (generalized) second derivative of $\ell$. 
Thanks to the $\ell_2$ regularization, the Hessian is always positive definite (Exercise) and consequently $f$ is a strongly convex function. In particular, we can apply TNCG without worrying for negative curvature directions (basically removing the control in Step 5).
The Newton inequality is solved with a truncation, because the amount of features $n$ is usually very large (e.g., text data with millions of tokens) making it not viable load the Hessian in a RAM. In fact, let's say $n=50 \cdot 10^6$, then the matrix would need $n^2=2.5\cdot 10^{15}$ float which are each 32 bits = 4 bytes, for a total of $10 \cdot 10^{16}$ bytes= 10000 Terabytes (divided by 2 because the Hessian is symmetric).
On the other hand, the hessian can be computed by loading a single example (or a mini-batch of them) at the time, i.e., 
$$ \hess(w) = \Id + \lambda\sum_{i=1}^{M} D_{ii}(w) x_ix_i^T.$$
Moreover, CG only requires Hessian-vector products meaning that the whole Hessian is never explicitly formed (nor stored), only the $n$-dimensional vector $\hess(w)s$ (and their partial computations) are, i.e., 
$$ \hess(w)s = s + \lambda\sum_{i=1}^{M} D_{ii}(w) x_i(x_i^Ts).$$
\end{example}

\section{Trust-Region Methods}
This method is introduced to overcome the requirement of the pure Newton method of dealing with Hessian that are (at least in the neighborhood of $x^*$) positive definite. In fact,
\begin{equation*}
	\min_{x \in \mathbb{R}^n} q_k(x):= f(x_k) + \grad(x_k)^T (x - x_k) + \frac{1}{2}(x - x_k)^T \hess(x_k)(x - x_k).
\end{equation*}
may not admit a minimum otherwise. In order to take into account this issue, a suitable strategy could be that of performing the minimization of $q_k(s_k)$ on a neighborhood of $x_k$, i.e., 
\begin{equation}\label{eq:trust-region}
	\min_{\|s_k\| \leq \Delta_k} q_k(s_k), \quad \with s_k:=x-x_k.
\end{equation}
Notice the notation switch: in this and in the following section we will use $s_k$ instead of $d_k$, as these vectors are steps (they include the step size) and not directions (the step size is excluded). The radius $\Delta_k$ defining the spherical region around $x_k$ is usually determined in such a way that $f(x_{k+1})<f(x_k)$ and that the reduction of $f$ is close to that of the
quadratic model $q_k(s_k)$. In this way, the radius defines the region where the model can be considered reliable, i.e., the so-called trust region. 

\begin{algorithm}[H]\label{alg:trust-region}
	\caption{Trust-Region}
	
	\KwIn{Pick $x_0\in \Rn$ arbitrarly, $\epsilon>0, 0<\eta_1\leq \eta_2<1$ and $ 0 < \gamma_1< 1 < \gamma_1$.}
	
	$k = 0$
	
	\While{$||\grad(x_k)||>\epsilon$}{
		
		Compute a step $s_k$ as a solution to the problem \eqref{eq:trust-region}
		
		Compute $\rho_k:= \frac{f(x_k)-f(x_k+s_k)}{q_k(0)-q_k(s_k)}$
		
		\If{$\rho_k\geq \eta_1$}{$x_{k+1} = x_k +s_k$}
		
		\Else{$x_{k+1} = x_k$}
		
		$\Delta_{k+1} = \begin{cases}
			\gamma_2 \Delta_k \quad &\text{if } \rho_k\geq \eta_2\\
			\Delta_k \quad &\text{if } \rho_k\in [\eta_1, \eta_2)\\
			\gamma_1 \Delta_k \quad &\text{if } \rho_k<\eta_1\\
		\end{cases}$
		
		$k = k+1$
	}
\end{algorithm}
\noindent In order to study the convergence of Algorithm \ref{alg:trust-region}, we will simplify it. First of all we will allow $\hess(w_k)$ to be replaced by $B_k$, an approximation of the Hessian. We call $m_k$ the corresponding model. 
\begin{equation}\label{eq:first-order_tr}
	\min_{\|s_k\| \leq \Delta_k} m_k(s_k):= f(x_k) + \grad(x_k)^Ts_k + \frac{1}{2}s_k^T B_ks_k.
\end{equation}
Second, instead of  solving \eqref{eq:first-order_tr} exactly, we will solve it by finding a step $s_k$ such that 
\begin{equation}\label{eq:cauchy_point}
	m_k(s_k) \leq m_k(s_k^C), \quad \with s_k^C:= -t_k^C\grad(x_k), \quad s_k^C \in \argmin_{0\leq t\leq \frac{\Delta_k}{\|\grad(x_k)\|}} m_k(-t\grad(x_k)),
\end{equation}
where $s_k^C$ is called Cauchy step, achieving the best decrease possible along the anti-gradient, and $x_k^C:=x_k + s_k^C$ is called Cauchy point. The corresponding algorithm is reported here for completeness.

\begin{algorithm}[H]\label{alg:trust-region_C}
	\caption{Trust-Region with Cauchy point}
	
	\KwIn{Pick $x_0\in \Rn$ arbitrarly, $\epsilon>0, 0<\eta_1\leq \eta_2<1$ and $ 0 < \gamma_1< 1 < \gamma_1$.}
	
	$k = 0$
	
	\While{$||\grad(x_k)||>\epsilon$}{
		
		Compute a step $s_k$ that satisfies \eqref{eq:cauchy_point}. 
%	$m_k(s_k) \leq m_k(s_k^C), \quad \with s_k^C:= -t_k^C\grad(x_k), \quad s_k^C \in \argmin_{0\leq t\leq \frac{\Delta_k}{\|\grad(x_k)\|}} m_k(-t\grad(x_k)),$
		
		Compute $\rho_k:= \frac{f(x_k)-f(x_k+s_k)}{q_k(0)-q_k(s_k)}$
		
		\If{$\rho_k\geq \eta_1$}{$x_{k+1} = x_k +s_k$}
		
		\Else{$x_{k+1} = x_k$}
		
		$\Delta_{k+1} = \begin{cases}
			\gamma_2 \Delta_k \quad &\text{if } \rho_k\geq \eta_2\\
			\Delta_k \quad &\text{if } \rho_k\in [\eta_1, \eta_2)\\
			\gamma_1 \Delta_k \quad &\text{if } \rho_k<\eta_1\\
		\end{cases}$
		
		$k = k+1$
	}
\end{algorithm}
\noindent Classical values for the above constants are $\eta_1=0.01, \eta_2=0.9, \gamma_1=0.5, \gamma_1=2$. We will thus study the convergence of Algorithm \ref{alg:trust-region_C} fist. Let us start by defining 
\begin{equation*}
	S := \{k \in \mathbb{N} \mid \rho_k \geq \eta_1\} \quad \text{and} \quad \mathcal{U}:= \mathbb{N} \setminus S,
\end{equation*}
the sets of \textbf{successful} and \textbf{unsuccessful} iterations, respectively, and
\begin{equation*}
	S_k := \{j \in [k] \mid \rho_j \geq \eta_1\} \quad \text{and} \quad \mathcal{U}_k := [k] \setminus S_k,
\end{equation*}
the corresponding sets up to iteration $k$. Notice that $x_{k+1} = x_k + s_k$ for $k \in S$, while $x_{k+1} = x_k$ for $k \in \mathcal{U}$.

We now state a property of Algorithm \ref{alg:trust-region_C} that solely depends on the mechanism to update the trust-region radius (Step 9).

\begin{lemma}[Successful and unsuccessful trust-region iterations]
	Let $\{x_k\}$ and $\{\Delta_k\}$ be the sequences generated by Algorithm \ref{alg:trust-region_C}. Assume that $\exists \; \Delta_{\min} > 0: \Delta_k \geq \Delta_{\min}$. Then
	\begin{equation*}
		k \leq |S_k| \left( 1 + \frac{\log \gamma_2}{|\log \gamma_1|} \right) + \frac{1}{|\log \gamma_1|} \left| \log \left( \frac{\Delta_{\min}}{\Delta_0} \right) \right|.
	\end{equation*}
\end{lemma}

\begin{proof}
	Exercise, it can be derived by noticing that $\Delta_{\min} \leq \Delta_k \leq \Delta_0 \gamma_2^{|S_k|} \gamma_1^{|\mathcal{U}_k|}$ and that $k= |S_k| + |\mathcal{U}_k|$.
\end{proof}
\noindent The first crucial result is well known in the trust-region literature as the Cauchy
decrease property, and assesses the magnitude of the function decrease at each iteration. From now on we will assume that $\grad(x_k)\neq 0$, otherwise the algorithm would have terminated.  
\begin{lemma}[Model decrease at the Cauchy point]\label{lemma:model_decrease}
	Given $s_k^C$ a Cauchy point as defined in \eqref{eq:cauchy_point}, we have
	\begin{equation}\label{eq:model_decrease}
		m_k(0)-m_k(s_k^C) \geq \frac{1}{2}\|\grad(x_k)\|\cdot \min\{\Delta_k, \frac{||\grad(x_k)||}{\|B_k\|} \}
	\end{equation}
\end{lemma}
\begin{proof}
	Let us rewrite $m_k$ as a function of one variable, $t_k$, i.e., 
	$$\min_{\|s_k\| \leq \Delta_k} f(x_k) -t_k \|\grad(x_k)\|^2 + \frac{1}{2}t_k^2\grad(x_k)^T B_k\grad(x_k),$$
	and let us notice that this is a parabola convex or concave depending on the sign of $\grad(x_k)^T B_k\grad(x_k)$. 
	Case I: $\grad(x_k)^T B_k \grad(x_k) \leq 0$. The parabola is concave, meaning that it is strictly monotonically decreasing (along the gradient) for increasing values of $t > 0$, so that, the minimizer is the upper bound of the interval defining the feasible set, and hence we have
	\begin{equation*}
		t^*_k = \frac{\Delta_k}{\|\grad(x_k)\|}.
	\end{equation*}
Thus, 
\begin{align*}
	m_k(0) - m_k(s_k^C) & = \Delta_k \|\grad(x_k)\| - \frac{1}{2}{\left(t_k^C\right)}^2\grad(x_k)^T B_k\grad(x_k) \geq \Delta_k \|\grad(x_k)\|
\end{align*}
	Case II: $\grad(x_k)^T B_k \grad(x_k) > 0$ (convex parabola) with the vertex that lies outside the trust-region, i.e., 
	\begin{equation*}
		t^*_k = \frac{\Delta_k}{\|\grad(x_k)\|}
	\end{equation*}
Here, we use convexity of $m_k$, that is 
\begin{align*}
	m_k(s_k^C) & \geq m_k(0) + \nabla m_k(0)^T(s_k^C-0) = m_k(0) - t_k^*||\grad(x_k)\|^2,
\end{align*}
from which we get 
$$m_k(0) - m_k(s_k^C) \geq \Delta_k \|\grad(x_k)\|$$
Case III: $\grad(x_k)^T B_k \grad(x_k) > 0$ (convex parabola) with the vertex that lies inside the trust-region, i.e., 
$$t^*_k = \frac{\|\grad(x_k)\|^2}{\grad(x_k)^T B_k \grad(x_k)}.$$
Thus, 
\begin{align*}
	m_k(0) - m_k(s_k^C) &=  t_k \left(\|\grad(x_k)\|^2 - \frac{1}{2} \|\grad(x_k)\|^2 \right)\\
	& = \frac{1}{2} t_k \|\grad(x_k)\|^2\\
	&\geq \frac{1}{2} \|\grad(x_k)\|^2 \cdot \frac{1}{||B_k||}\geq \frac{1}{2} \|\grad(x_k)\|^2 \cdot \frac{1}{||B_k||}
\end{align*}
Putting all together, we get \eqref{eq:model_decrease}.
\end{proof}

\begin{lemma}[Condition for a successful iteration]\label{lemma:successful_condition}
	Let $\{x_k\}$ and $\{\Delta_k\}$ be the sequences generated by Algorithm \ref{alg:trust-region_C}. Suppose that $f\in \LC$, that $\exists \, L_B>0: \|B_k\| \leq L_B$ and that
	\begin{equation}\label{eq:successful_condition}
		\Delta_k < \delta\|\grad(x_k)\|, \quad \with \delta:=\frac{(1 - \eta_2)}{2(L + L_B)}
	\end{equation}
	Then $\rho_k \geq \eta_2$, iteration $k$ is successful, and $\Delta_{k+1} \geq \Delta_k$.
\end{lemma}

\begin{proof}
	We obtain from \eqref{eq:successful_condition} and $0 < \eta_2 < 1$ that
	\begin{equation*}
		\Delta_k < \frac{\|\grad(x_k)\|}{2(L + L_B)} \leq \frac{\|\grad(x_k)\|}{L_B}.
	\end{equation*}
	As a consequence, together with Lemma \ref{lemma:model_decrease} and \eqref{eq:cauchy_point}, we get
	\begin{align*}
		m_k(0) - m_k(s_k) &\geq m_k(0) - m_k(s_k^c) \\
		&\geq \frac{1}{2} \|\nabla_x f(x_k)\| \min \left[ \frac{\|\grad(x_k)\|}{L_B}, \Delta_k \right] \\
		&= \frac{1}{2} \|\grad(x_k)\| \Delta_k.
	\end{align*}
	Now, from the Mean Value Theorem, the Cauchy--Schwarz inequality, $f\in \LC(\Rn)$ and $\|B_k\| \leq L_B$ we get
	\begin{equation*}
		|f(x_k + s_k) - m_k(s_k)| = |\left(\grad(x_k+\xi_k s_k) - \grad(x_k) \right)^Ts_k  - \frac{1}{2}  s_k^TB_k s_k | \leq (L + L_B) \|s_k\|^2
	\end{equation*}
	for some $\xi_k \in [0, 1]$. Combining this relation with the inequality above and using the bound $\|s_k\| \leq \Delta_k$, we then obtain that
	\begin{equation*}
		|\rho_k - 1| = \left| \frac{f(x_k + s_k) - m_k(s_k)}{m_k(0) - m_k(s_k)} \right| \leq \frac{(L + L_B) \Delta_k^2}{\frac{1}{2} \|\grad(x_k)\| \Delta_k} \leq 1 - \eta_2
	\end{equation*}
	because of \eqref{eq:successful_condition}. Hence $\rho_k \geq \eta_2 \geq \eta_1$, $k \in S_k$, and the remaining conclusion follows from Step 9 of Algorithm \ref{alg:trust-region_C}. 
\end{proof} 

\begin{lemma}[Lower bound for the trust-region radius]\label{lemma:deltamin}
	Let $\{x_k\}$ and $\{\Delta_k\}$ be the sequences generated by Algorithm \ref{alg:trust-region_C}. Suppose that $f\in \LC$, that $\exists\, L_B>0: \|B_k\| \leq L_B$ and that $\Delta_0\geq \gamma_1 \delta \|\grad(x_0)\|$. Then we have that
	\begin{equation}\label{eq:deltamin}
		\Delta_k \geq \gamma_1 \delta \min_{i \in [0:k]} \|\grad(x_i)\| \quad \forall k\geq 0,
	\end{equation}
	where $\delta$ is defined in Lemma \ref{lemma:successful_condition}.
\end{lemma}
\begin{proof}
By assumption, \eqref{eq:deltamin} holds for $k = 0$. By contradiction, assume that iteration $k \geq 1$ is the first such that \eqref{eq:deltamin} fails. Since Step 9 of Algorithm \ref{alg:trust-region_C} ensures that $\gamma_1 \Delta_{k-1} \leq \Delta_k$, we deduce that
\begin{equation*}
	\Delta_{k-1} \leq \frac{1}{\gamma_1}\Delta_k < \delta \min_{i \in [0:k]} \|\grad(x_i)\|,
\end{equation*}
and consequently also
\begin{equation*}
	\frac{\Delta_{k-1}}{\|\grad(x_{k-1})\|} \leq \frac{\Delta_{k-1}}{\min_{i \in [0:k-1]} \|\grad(x_i)\|} \leq \frac{\Delta_{k-1}}{\min_{i \in [0:k]} \|\grad(x_i)\|} < \delta.
\end{equation*}
Hence Lemma \ref{lemma:successful_condition} guarantees that $\Delta_{k-1} \leq \Delta_k$, and therefore, from the fact that \eqref{eq:deltamin} is violated at iteration $k$, we get
\begin{equation*}
	\frac{\Delta_{k-1}}{\min_{i \in [0:k-1]} \|\grad(x_i)\|} \leq \frac{\Delta_k}{\min_{i \in [0:k]} \|\grad(x_i)\|} < \gamma_1 \delta.
\end{equation*}
As a consequence, \eqref{eq:deltamin} is also violated at iteration $k - 1$. But this contradicts the assumption that iteration $k$ is the first such that this inequality fails. 
\end{proof}
\noindent Notice that the assumption $\Delta_0\geq \gamma_1 \delta \|\grad(x_0)\|$ can be removed, but the corresponding analysis only gets computational heavier.

\begin{lemma}[Model decrease]\label{lemma:model_decrease2}
	Let $\{x_k\}$ and $\{\Delta_k\}$ be the sequences generated by Algorithm \ref{alg:trust-region_C}. Suppose that $f\in \LC$, that $\exists\, L_B>0: \|B_k\| \leq L_B$ and that $\Delta_0\geq \gamma_1 \delta \|\grad(x_0)\|$. Then we have that, for all $k \geq 0$,
	\begin{equation*}
		m_k(0) - m_k(s_k) \geq c_1 \|\grad(x_k)\| \min_{i \in [0:k]} \|\grad(x_i)\|,
	\end{equation*}
	where
	\begin{equation*}
		c_1 := \frac{1}{2} \min \left[ \frac{1}{L_B}, \gamma_1\delta \right] \quad \with \delta \text{ defined in Lemma \ref{lemma:successful_condition}}.
	\end{equation*}
\end{lemma}

\begin{proof}
	Exercise, it follows from \eqref{eq:cauchy_point}, Lemma \ref{lemma:model_decrease} and Lemma \ref{lemma:deltamin}.
\end{proof}

\noindent We are now in a position to bound the total number of successful iterations necessary to find an $\epsilon$-approximate first-order minimizer using Algorithm \ref{alg:trust-region_C}.

\begin{lemma}[Bound on the number of successful iterations of Algorithm \ref{alg:trust-region_C}]
	Assume that $f\in \LC$, that $\exists\, L_B>0: \|B_k\| \leq L_B$ and that $\Delta_0\geq \gamma_1 \delta \|\grad(x_0)\|$. Additionally, let $f$ be lower bounded by $f^*$. Then there exists a positive constant $c_2$ such that, for any $\epsilon > 0$, Algorithm \ref{alg:trust-region_C} requires at most $c_2 \frac{f(x_0) - f_{\text{low}}}{\epsilon^2}$
%	\begin{equation*}
%		K_S:= c_2 \frac{f(x_0) - f_{\text{low}}}{\epsilon^2}
%	\end{equation*}
	successful iterations before an iterate $x_{k_\epsilon+1}$ is computed for which $\|\grad(x_{k_\epsilon+1})\| \leq \epsilon$.
\end{lemma}
\begin{proof}
First recall that if $j\in \mathcal{U}$, we have $f(x_j) = f(x_{j+1})$ and $\grad(x_j)=\grad(x_{j+1})$. As $k_\epsilon+1$ is the first iteration for which $\|\grad(x_{k_\epsilon+1})\| \leq \epsilon$, we have $\|\grad(x_k)\|>\epsilon \; \forall \, k <k_\epsilon+1$. This, together with the fact that $f(x)\geq f^*$ and Lemma \ref{lemma:model_decrease2}, by telescoping sum brings to 
\begin{align*}
	f(x_0)-f^* &\geq f(x_0) - f(x_{k_\epsilon+1})\\
	& =\sum_{j\in S_{k_\epsilon}} f(x_j) - f(x_{j+1})\\
	& \geq \eta_1 (m_j(0)- m_j(s_j))\\
	& \geq \eta_1 c_2 |S_{k_\epsilon}| \epsilon^2,
\end{align*}
where the second inequality follows from Step 5 of Algorithm \ref{alg:trust-region_C}. Thus, $|S_k|\leq c_2 \frac{f(x_0) - f_{\text{low}}}{\epsilon^2}.$
\end{proof}

\begin{theorem}
	Assume that $f\in \LC$, $f$ be lower bounded by $f^*$, that $\exists\, L_B>0: \|B_k\| \leq L_B$ and that $\Delta_0\geq \gamma_1 \delta \|\grad(x_0)\|$. Algorithm \ref{alg:trust-region_C} has a iteration complexity of $\BigO(\epsilon^{-2})$.
\end{theorem}
\begin{proof}
	Exercise, by computing the precise constants.
\end{proof}
Now that we have the convergence rate of Algorithm \ref{alg:trust-region_C}, let us move back to the convergence analysis of Algorithm \ref{alg:trust-region} and see whether employing directly the Hessian, instead of $B_k$ will allow us to achieve a better rate. Unfortunately, the Trust-Region algorithm (Algorithm \ref{alg:trust-region}) inherits the same slow convergence of Newton. In fact, we have the following lower bound.

\begin{theorem}\label{thm:slow_trust_region}
	Algorithm \ref{alg:trust-region} applied to minimize a function $f\in \LC(\Rn)$ bounded from below may require as many as $\epsilon^{-2}$ evaluations of $f$ and $\grad$ to produce an iterate $x\in \Rn$ such that $||\grad (x)\|\leq \epsilon.$
\end{theorem}
\begin{proof}
	Exercise, one needs to prove that the sequence of iterates in Theorem \ref{thm:slow_newton} can be generated also by Algorithm \ref{alg:trust-region}. In particular, the function defined in Theorem \ref{thm:slow_newton} behaves quadratically around the iterates, yielding decreases in $f$ that fully align ($\rho_k=1$) with those of the quadratic model employed by the trust-region method.
\end{proof}
Given this result, there is no need to solve $q_k(s)$ to higher precision than that achieved by the Cauchy step (at least when concerned with the global speed of convergence). In particular, we can accept a step $s_k$ such that 
\begin{equation*}
	q_k(s_k) \leq q_k(s_k^C), \quad \with s_k^C:= -t_k^C\grad(x_k), \quad s_k^C \in \argmin_{0\leq t\leq \frac{\Delta_k}{\|\grad(x_k)\|}} m_k(-t\grad(x_k)),
\end{equation*}
where the only difference with \eqref{eq:cauchy_point} is that we use $\hess(x_k)$ instead of $B_k$. For this algorithm, and consequently also for Algorithm \ref{alg:trust-region} the convergence speed can be inherited from that of Algorithm \ref{alg:trust-region_C} when we notice that $\hess(x_k)$ is a special case of $B_k$. For this result, the assumption that $\|B_k\|\leq L_B$ is not needed, as the Hessian is upper bounded if and only if $f\in \LC(\Rn)$, which was already a requirement of the results above. 

Concerning the local convergence, if $\{x_k\}$ converges towards $x^*$ and $\hess(x^*)\succ0$, Algorithm \ref{alg:trust-region} will achieve local quadratic convergence, as the resulting steps around $x^*$ will be the same of pure Newton. In fact, similarly to the line search-based globalization of Newton, the quadratic model will be accurate and the trust-region radius will be always doubled (multiplied by $\gamma_2$).
\bibliographystyle{plain}
\bibliography{../biblio}
\end{document}