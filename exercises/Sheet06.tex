\documentclass{ExerciseSheet}

%Set Number of the Exercise sheet and the submission deadline.
\setExerciseSheetNumber{6}
\setSubmissionDate{xx.xx.2024}

%boolean variable to determine whether the solutions should be included
\newif\ifsolutions
\solutionstrue
%\solutionsfalse

%We have a figure in this sheet
\usepackage{graphicx}
\usepackage{enumitem}


\begin{document}


%Start with exercises
%-----------------------------------------------------------------------%



%-----------------------------------------------------------------------%
\begin{problem}
In this problem we want to work on the exact line search. 

As a reminder exact line search is used to determine the step size in the Descent Direction Method.  
So suppose we have a function $f:\R^n \rightarrow \R$ and a time step $k$ and a descent direction $d_k$, then we choose the step size as the minimizer along this direction,  i.e.
\begin{equation*}
    t_k \in \argmin_{t>0} f(x_k+td_k)
\end{equation*}
\renewcommand{\labelenumi}{\alph{enumi})}
We start with an example. 
\begin{enumerate}
    \item Let $A\in \R^{n\times n}$ be positive definite, $b\in \R^n$ and $c\in \R$. Define the function $f:\R^n \rightarrow \R$ by $$f(x)=x^TAx + 2b^Tx +c .$$ Let $d\in \R^n$ be a descent direction of $f$ at a point $x$. What is the result of the exact line search? 
\end{enumerate}
\noindent Now we do the analytical analysis.
\begin{enumerate}[resume]
    \item Let $f\in C^{1,1}_L(\R^n)$ and $(x_k)_{k\in \N}$ be a sequence generated by the gradient method for solving $\min_{x\in \R^n} f(x)$ with the exact line search determining the stepsize. \\
    Show that \begin{equation*}
        f(x_k) - f(x_{k+1}) \geq \frac{1}{2L} \|\nabla f(x_k)\|_2^2.
    \end{equation*}
\end{enumerate}
\end{problem}
\ifsolutions
\vskip 0.3cm
\begin{solution}
    \renewcommand{\labelenumi}{\alph{enumi})}

\begin{enumerate}
    \item Set $g:\R \rightarrow \R$, $g(t)=f(x+td)$. Now the goal is to minimizie (the one dimensional function) g. We reformulate
    \begin{align*}
        g(t) &= (x+td)^TA (x+td) + 2b^T(x+td) + c \\
        &= t^2d^TAd + 2t \left(d^TAx + b^Td\right) + x^TAx + 2b^Tx + c \\
        &=t^2d^TAd + 2t \left(d^TAx + b^Td\right) + f(x)
    \end{align*}
    Now 
    \begin{align*}
        g'(t)&= 2d^TAd t + 2\left(d^TAx + b^Td\right).
    \end{align*}
    Setting the derivative to zero yields
    \begin{equation}
        g'(t) = 0 \Leftrightarrow t = -\frac{d^TAx + b^Td}{d^TAd},
    \end{equation}
    where we use that $A$ is positive definite and $d\neq 0$
    Question: Why is $t>0$, descent direction!! \\ 
    Now we calculate the second derivative
    \begin{equation*}
        g''(t)=2d^TAd >0.
    \end{equation*}
    \textbf{Also discuss when the solution is unique etc in the tutorial!!}
    \item By the Descent Lemma (Lemma 2.3 in the lecture notes) we know
    \begin{equation*}
        f(x)-f(x-t\nabla f(x)) \geq t\left(1-\frac{Lt}{2}\right) \|\nabla f(x)\|^2, \quad \forall t\geq 0.
    \end{equation*}
    Let $t_k \in \argmin_{t>0} f(x_k +t\nabla f(x_k))$, and define then $x_{k+1}=x_k - t_k \nabla f(x_k)$. By the definition of $t_k$ we have $f(x_{k+1})\leq f(x_k - \frac{1}{L}\nabla f(x_k))$ and hence
    \begin{equation*}
        f(x_k) - f(x_{k+1}) \geq f(x_k) - f(x_k - \frac{1}{L}\nabla f(x_k)) \geq \frac{1}{L}(1-\frac{1}{2}) \|\nabla f(x_k)\|^2.
    \end{equation*}
 \end{enumerate}
\end{solution}

\fi


%-----------------------------------------------------------------------%
\vskip 0.5cm

\begin{exo}
Let $\{x_k\}_{k\in \N}$ be a sequence generated by gradient method with exact line search for solving a problem of minimizing a continuously differentiable function f. Then for any $k$
\begin{equation*}
    \left(x_{k+2} - x_{k+1}\right)^T\left(x_{k+1}-x_k\right) = 0
\end{equation*}
\end{exo}

\ifsolutions
\vskip 0.3cm
\begin{solution}
We have $x_{k+1}-x_k=-t_k \nabla f(x_k)$ for all $k$, hence the above simplifies to 
\begin{equation*}
    t_{k+1}t_k\nabla f(x_{k+1})^T \nabla f(x_k) = 0,
\end{equation*}
which is the case (since $t_k>0$) iff 
\begin{equation*}
   \nabla f(x_{k+1})^T \nabla f(x_k) = 0,
\end{equation*}
We know $t_k \in \argmin_{t > 0} g(t):=f(x_k - t \nabla f(x_k))$, so $0= g'(t_k)= -\nabla f(x_k)^T \nabla f(x_k - t_k \nabla f(x_k))= - \nabla f(x_k)^T \nabla f(x_{k+1})$
\end{solution}

\fi

%-----------------------------------------------------------------------%
\vskip 0.5cm
\begin{exo}
The goal of this exercise is to prove Gradient Descent with Armijo Line Search even when we only have $f\in C^1.$

So assume $f: \R^n \rightarrow \R$ is in $C^1$ and lower bounded. Then Armijo's method terminates and determines a step-size $t_k$ s.t. 
\begin{enumerate} 
    \item $f(x_{k+1}) < f(x_k)$,
    \item Every accumulation point of $(x_k)_{k \in \N}$ is a stationary point of $f$.
\end{enumerate}
Hint: Use the mean-value-theorem: For $a<b$ there exists a $c \in (a,b)$ s.t. $f(b) - f(a) = (b-a) f'(c)$.
\end{exo}


\ifsolutions
\vskip 0.3cm
\begin{solution}
First part is clear. \\
Why does the Armijo step size rule always terminate?? \\

1. Show $f(x_k)$ converges \\
 2. Show $t_k\|\nabla f(x_k)\|^2$ converges to zero. \\
3. Assume the sequence does not converge to a stationary point
\end{solution}
\fi

%-----------------------------------------------------------------------%
\vskip 0.5cm
\begin{exo}
	Consider the quadratic minimization problem $f(x)=x^TAx$, where $A\in \R^{5 \times 5}$ and $A_{i,j}=\frac{1}{i+j-1}$.
 Run Gradient descent with different step size update rules, and initial vector $x_0=(1,2,3,4,5)^T$.
\end{exo}

\ifsolutions
\vskip 0.3cm
\begin{solution}
\end{solution}
\fi

\end{document}