\documentclass[10pt,a4paper]{article}
\usepackage[utf8]{inputenc}
\usepackage[T1]{fontenc}
\usepackage{amsmath}
\usepackage{amsthm}
\usepackage{amssymb}
\usepackage{graphicx}
\usepackage{mathtools}
\usepackage[ruled,vlined,linesnumbered]{algorithm2e}
\usepackage[left=2.50cm, right=2.50cm, top=2.0cm, bottom=2.0cm]{geometry}


\makeatother
\DeclareMathOperator*{\argmin}{argmin}
\DeclareMathOperator*{\Max}{\text{max}}
\DeclareMathOperator*{\E}{\mathbb{E}}
\newcommand{\Ei}[1]{\mathbb{E}_{#1}}
\DeclareMathOperator*{\Eik}{\mathbb{E}_{\mathit{i_k}}}
\DeclareMathOperator*{\Eikplus}{\mathbb{E}_{\mathit{i_{k+1}}}}
\newcommand{\Eiplus}[1]{\mathbb{E}_{i_{#1}}}
\DeclareMathOperator*{\LC}{\text{\textup{LC}}^1}
\DeclareMathOperator*{\grad}{\mathit{\nabla \!f}}
\DeclareMathOperator*{\argmax}{arg\;max}
\DeclareMathOperator*{\gradik}{\mathit{\nabla\!\fik}}
\DeclareMathOperator*{\Lmax}{\mathit{L_{max}}}
\newcommand{\R}{\mathbb{R}}
\newcommand{\N}{\mathbb{N}}
\newcommand{\C}{\mathbb{C}}
\newcommand{\diag}{\text{diag}}
\newcommand{\interior}{\text{int}}
\newcommand{\with}{\text{with }\,}
\newcommand{\Tr}{\text{Tr}}
\newcommand{\st}{\text{s.t.} \;\;\;}
\newcommand{\Rn}{\mathbb{R}^n}
\newcommand{\Rnn}{\mathbb{R}^{n\times n}}
\newcommand{\Rmn}{\mathbb{R}^{m\times n}}
\newcommand{\ikplus}{i_{k+1}}
\newcommand{\wrefi}[2]{w_{r(#1, i_{#2})}}
\newcommand{\wref}[1]{\wrefi{#1}{#1}}
\newcommand{\fik}{f_{i_k}}
\newcommand{\fikofwstar}{\fik(w^*)}
\newcommand{\fofwstar}{f(w^*)}
\newcommand{\fii}[1]{f_{i_{#1}}}
\newcommand{\fikplus}{f_{i_{k+1}}}
\newcommand{\fiplus}[1]{f_{i_{#1}}}
\newcommand{\fimax}{f_i^{\text{max}}}
\newcommand{\fikmax}{\fik^{\text{max}}}
\newcommand{\bmax}{b_{\text{max}}}
\newcommand{\fmax}{f^{\text{max}}}
\newcommand{\fmaxk}[1]{f_{i_{#1}}^{\text{max}}}
\newcommand{\Lik}{L_{i_k}}
\newcommand{\Cik}{C_{i_k}}
\newcommand{\deltak}{\delta^{l_k}}
\newcommand{\deltakplus}{\delta^{l_k+1}}
\newcommand{\deltakminus}{\delta^{l_k-1}}
\newcommand{\etatilde}{\tilde{\eta}_{k,0}}
\newcommand{\muik}{\mu_{i_k}}
\newcommand{\etamax}{\eta^{\text{max}}}
\newcommand{\etamaxx}{\bar{\eta}^{\text{max}}}
\newcommand{\etamin}{\eta^{\text{min}}}
\newcommand{\etaminn}{\bar{\eta}^{\text{min}}}
\newcommand{\minimum}[2]{\min \left\{ #1, #2 \right \} }
\newcommand{\maximum}[2]{\max \left\{ #1, #2 \right \} }
\newcommand{\W}[1]{{\scriptscriptstyle W #1}}
\newcommand{\gradi}[1]{\nabla f_{i_{#1}} (w_{i_{#1}}) }
\newcommand{\Id}{\text{Id}}
\newcommand{\inprod}[2]{\langle #1, #2 \rangle}
\makeatletter

\newtheorem{assumption}{Assumption}
\newtheorem{lemma}{Lemma}
\newtheorem{proposition}{Proposition}
\newtheorem{theorem}{Theorem}
\newtheorem{corollary}{Corollary}
\newtheorem{remark}{Remark}
\newtheorem{example}{Example}
\newtheorem{definition}{Definition}


\newcommand{\imgS}{.26}
\newcommand{\dir}{exp1/}
\newcommand{\model}{mlp}
\newcommand{\modelname}{mlp}


\newcommand{\mlp}{{\texttt{mnist|mlp}}}
\newcommand{\res}{{\texttt{cifar10|resnet34}}}
\newcommand{\dense}{{\texttt{cifar10|densenet121}}}
\newcommand{\ress}{{\texttt{cifar100|resnet34}}}
\newcommand{\denses}{{\texttt{cifar100|densenet121}}}
\newcommand{\fashion}{{\texttt{fashion|effb1}}}
\newcommand{\svhn}{{\texttt{svhn|wrn}}}
\newcommand{\wiki}{{\texttt{wiki2|encoder}}}
\newcommand{\ptb}{{\texttt{ptb|xl}}}
\newcommand{\mushrooms}{{\texttt{mushrooms}}}
\newcommand{\rcvone}{{\texttt{rcv1}}}
\newcommand{\ijcnn}{{\texttt{ijcnn}}}
\newcommand{\weighta}{{\texttt{w8a}}}

\title{Continuous Optimization}
\author{Chapter 2: Gradient Descent}
\date{}
\begin{document}
	\maketitle
	\section{Descent Direction Methods}
	\noindent In this chapter we consider the unconstrained minimization problem
	\begin{equation*}
		\min_{x\in\Rn} \;\; f(x).
	\end{equation*}
The iterative algorithms that we will consider in this chapter take the form
\begin{equation*}
	x_{k+1} = x_k +t_k d_k \quad k=0,1, \dots,
\end{equation*}
where $d_k$ is the so-called direction and $t_k$ is the step size. We will limit ourselves to descent
directions, whose definition is now given.
\begin{definition}
	Let $f:\Rn\to\R$ with $f\in\C(\Rn)$. A vector $0\neq d\in \Rn$ is called a descent direction of $f$ if the directional derivative $f'(x,d)$ is negative, i.e., 
	\begin{equation*}
		f'(x,d)= \grad(x)^T d < 0.
	\end{equation*}
\end{definition}
In particular, by taking small enough steps, descent directions lead to a decrease of the objective function.
\begin{lemma}[descent property of descent directions]
	Let $f:\Rn\to\R$ with $f\in\C(\Rn)$ and let $x\in \Rn$. Suppose that $d$ is a descent direction of $f$ at $x$. Then, there exists $\epsilon>0$ such that 
	\begin{equation*}
		f(x+td) < f(x) \quad \forall \, t \in (0,\epsilon].
	\end{equation*} 
\end{lemma}
\begin{proof}
	Since $f'(x,d)<0$, it follows from the definition of the directional derivative that 
	\begin{equation*}
		\lim_{t\to 0^+}\frac{f(x+td)-f(x)}{t} = f'(x,d) <0.
	\end{equation*}
Therefore, there exists an $\epsilon>0$ such that 
\begin{equation*}
	\frac{f(x+td)-f(x)}{t}<0,
\end{equation*}
for any $t\in(0,\epsilon)$
\end{proof}
\begin{algorithm}[H]\label{alg}
	\caption{Schematic Descent Directions Method}
	
	\KwIn{$x_0\in \Rn$}
	
	$k = 0$
	
	\While{Termination criterion is not satisfied}{
				
			Pick a descent direction $d_k$
			
			Find a step size $t_k$ satisfying $f(x_k+t_kd_k)<f(x_k)$
			
			$x_{k+1} = x_k+t_kd_k$
			
			$k = k+1$
		}
\end{algorithm}
\noindent Various are still unspecified.
\section{Gradient Method}
\noindent The most important choice in the algorithm above concerns the selection of the descent direction. One obvious choice is to pick the steepest (normalized) direction, i.e., $d_k =-\grad(x_k)/||\grad(x_k)||$. In fact, this direction minimizes the directional derivatives between all normalized directions. 
\begin{lemma}
	Let $f:\Rn\to\R$ with $f\in\C(\Rn)$ and let $x\in\Rn$ be non-stationary (i.e., $\grad(x)\neq0$). Then the optimal solution of the problem
	\begin{equation*}
		\begin{split}
			\min \;\; &f'(x,d),\\
			\st& ||d||=1.
		\end{split}
	\end{equation*}
is $d=-\frac{\grad(x)}{||d||}$.
\end{lemma}
\begin{proof}
	As $f\in \C(\Rn)$ and by Cauchy-Schwarz, we have 
	\begin{equation*}
		f'(x,d)=\grad(x)^Td \geq -||\grad(x)||\cdot ||d|| = -||\grad(x)||.
	\end{equation*}
Thus, $-||\grad(x)||$ is a lower bound for the optimal value of the problem. On the other hand, by plugging  $d = -\grad(x)/||\grad(x)||$ in the objective function we get 
\begin{equation*}
	f'\left(x,-\frac{\grad(x)}{||\grad(x)||}\right)=-\grad(x)^T\left(\frac{\grad(x)}{||\grad(x)||}\right)= -||\grad(x)||,
\end{equation*}
and we thus come to the conclusion that the lower bound is attained at $d=-\frac{\grad(x)}{||d||}$.
\end{proof}
\noindent Thus, the gradient method selects $d_k = -\grad(x_k)$ which is obviously a descent direction, i.e., 
\begin{equation*}
\grad(x_k)^Td_k = -\grad(x_k)^T\grad(x_k) = -||\grad(x)||^2.
\end{equation*}

\bibliographystyle{plain}
\bibliography{../biblio}
\end{document}