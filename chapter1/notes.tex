\documentclass[10pt,a4paper]{article}
\usepackage[utf8]{inputenc}
\usepackage[T1]{fontenc}
\usepackage{amsmath}
\usepackage{amsthm}
\usepackage{amssymb}
\usepackage{graphicx}
\usepackage{mathtools}
\usepackage[left=2.50cm, right=2.50cm, top=2.0cm, bottom=2.0cm]{geometry}


\makeatother
\DeclareMathOperator*{\argmin}{argmin}
\DeclareMathOperator*{\Max}{\text{max}}
\DeclareMathOperator*{\E}{\mathbb{E}}
\newcommand{\Ei}[1]{\mathbb{E}_{#1}}
\DeclareMathOperator*{\Eik}{\mathbb{E}_{\mathit{i_k}}}
\DeclareMathOperator*{\Eikplus}{\mathbb{E}_{\mathit{i_{k+1}}}}
\newcommand{\Eiplus}[1]{\mathbb{E}_{i_{#1}}}
\DeclareMathOperator*{\LC}{\text{\textup{LC}}^1}
\DeclareMathOperator*{\grad}{\mathit{\nabla \!f}}
\DeclareMathOperator*{\argmax}{arg\;max}
\DeclareMathOperator*{\gradik}{\mathit{\nabla\!\fik}}
\DeclareMathOperator*{\Lmax}{\mathit{L_{max}}}
\newcommand{\R}{\mathbb{R}}
\newcommand{\st}{\text{s.t.} \;\;\;}
\newcommand{\Rn}{\mathbb{R}^n}
\newcommand{\ikplus}{i_{k+1}}
\newcommand{\wrefi}[2]{w_{r(#1, i_{#2})}}
\newcommand{\wref}[1]{\wrefi{#1}{#1}}
\newcommand{\fik}{f_{i_k}}
\newcommand{\fikofwstar}{\fik(w^*)}
\newcommand{\fofwstar}{f(w^*)}
\newcommand{\fii}[1]{f_{i_{#1}}}
\newcommand{\fikplus}{f_{i_{k+1}}}
\newcommand{\fiplus}[1]{f_{i_{#1}}}
\newcommand{\fimax}{f_i^{\text{max}}}
\newcommand{\fikmax}{\fik^{\text{max}}}
\newcommand{\bmax}{b_{\text{max}}}
\newcommand{\fmax}{f^{\text{max}}}
\newcommand{\fmaxk}[1]{f_{i_{#1}}^{\text{max}}}
\newcommand{\Lik}{L_{i_k}}
\newcommand{\Cik}{C_{i_k}}
\newcommand{\deltak}{\delta^{l_k}}
\newcommand{\deltakplus}{\delta^{l_k+1}}
\newcommand{\deltakminus}{\delta^{l_k-1}}
\newcommand{\etatilde}{\tilde{\eta}_{k,0}}
\newcommand{\muik}{\mu_{i_k}}
\newcommand{\etamax}{\eta^{\text{max}}}
\newcommand{\etamaxx}{\bar{\eta}^{\text{max}}}
\newcommand{\etamin}{\eta^{\text{min}}}
\newcommand{\etaminn}{\bar{\eta}^{\text{min}}}
\newcommand{\minimum}[2]{\min \left\{ #1, #2 \right \} }
\newcommand{\maximum}[2]{\max \left\{ #1, #2 \right \} }
\newcommand{\W}[1]{{\scriptscriptstyle W #1}}
\newcommand{\gradi}[1]{\nabla f_{i_{#1}} (w_{i_{#1}}) }
\makeatletter

\newtheorem{assumption}{Assumption}
\newtheorem{lemma}{Lemma}
\newtheorem{proposition}{Proposition}
\newtheorem{theorem}{Theorem}
\newtheorem{corollary}{Corollary}
\newtheorem{remark}{Remark}


\newcommand{\imgS}{.26}
\newcommand{\dir}{exp1/}
\newcommand{\model}{mlp}
\newcommand{\modelname}{mlp}


\newcommand{\mlp}{{\texttt{mnist|mlp}}}
\newcommand{\res}{{\texttt{cifar10|resnet34}}}
\newcommand{\dense}{{\texttt{cifar10|densenet121}}}
\newcommand{\ress}{{\texttt{cifar100|resnet34}}}
\newcommand{\denses}{{\texttt{cifar100|densenet121}}}
\newcommand{\fashion}{{\texttt{fashion|effb1}}}
\newcommand{\svhn}{{\texttt{svhn|wrn}}}
\newcommand{\wiki}{{\texttt{wiki2|encoder}}}
\newcommand{\ptb}{{\texttt{ptb|xl}}}
\newcommand{\mushrooms}{{\texttt{mushrooms}}}
\newcommand{\rcvone}{{\texttt{rcv1}}}
\newcommand{\ijcnn}{{\texttt{ijcnn}}}
\newcommand{\weighta}{{\texttt{w8a}}}

\title{Continuous Optimization}
\author{Chapter 1: Basics of Optimization}
\date{}
\begin{document}
	\maketitle
	\section{Optimization problems}
	In this course we will focus on finite-dimensional continuous optimization problems
	\begin{equation}\label{eq:problem}
	\begin{split}
		\min \;\; &f(x)\\
		\st&g_i(x) \leq 0 \quad i=1, \dots, m\\
		& h_j(x) = 0 \quad j=1, \dots, p
	\end{split}
	\end{equation}
\begin{itemize}
	\item $x \in \Rn$: vector of variables to be chosen ($n$ scalar variables $x_1 , \dots , x_n$)
	\item $f$: objective function to be minimized
	\item $g_1,\dots, g_m$: inequality constraint functions to be satisfied
	\item $h_1,\dots, h_p$: equality constraint functions to be satisfied
\end{itemize}
The feasible set can also be denoted by $S=\{x \in \Rn: g_i(x) \leq 0 \; i=1, \dots, m, \;\; h_j(x) = 0 \; j=1, \dots, p\}$.
\subsection{Classification}
Concerning the dimensionality, optimization problems can be 
\begin{itemize}
	\item infinite-dimensional: when the space of variables is infinite dimensional, e.g., elements can be vectors of functions of one ore more variables;
	\item finite-dimensional: when the space of variables is finite-dimensional, e.g., $\Rn.$
\end{itemize}
Concerning the granularity of the space, optimization problems can be 
\begin{itemize}
	\item discrete: when at least one variable is an integer;
	\item continuous: when all variables involved are continuous.
\end{itemize}
Concerning the hypothesis of linearity, optimization problems are also called
\begin{itemize}
\item linear programming: when the objective function is linear and the
feasible set is defined by a system of linear equalities and inequalities;
\item nonlinear problems: when the objective function is nonlinear
and/or at least one constraint is defined by a nonlinear function.
\end{itemize}
Concerning the hypothesis of differentiability, we can distinguish between
\begin{itemize}
\item smooth optimization: when the functions defining the objective
and the constraints are continuously differentiable;
\item nonsmooth optimization problems: when the functions defining the objective and the constraints are continuous, but typically not differentiable in all points of the feasible set. Weaker notions of derivatives are used to solve this problem.
\item derivative-free optimization: this branch of optimization consider cases in which evaluating the objective function/constraints is very costly, in particular the functions defining the objective and the constraints might not be continuous. A similar field is that of black-box optimization, in which the optimizer assumes to not have any information on the objective function or on the constraints.
\end{itemize}
Concerning the hypothesis of convexity, we can distinguish between
\begin{itemize}
	\item convex optimization, when the functions defining the objective and the constraints are all convex;
	\item nonconvex optimization, when the functions defining the objective and the constraints might be nonconvex.
\end{itemize}

\begin{example}[The transportation problem (Linear Programming)]
Consider a company producing and selling a certain commodity. The company has a set of $N$ sources, where the commodity is produced and a set of $M$ demand centers where the commodity is sold. At each source $i \in \{1, \dots, N\}$, a given quantity $q_i$ of commodity is available. Each demand center $j \in \{1, \dots, M\}$ requires a given quantity $d_j$. We indicate by $c_{ij}$ the cost per unit of transporting the commodity from source $i$ to demand center $j$.

\par The problem is to determine the quantity to be transported from each source to
each demand center in such a way that
\begin{itemize}
	\item the availability constraints at the sources are satisfied;
	\item the requirements of the demand centers are satisfied;
	\item the total transportation cost is minimized.
\end{itemize}
To formulate the problem, we indicate by $x_{ij}$ the quantity of the commodity transported from source $i$ to demand center $j$. Then the optimization problem is

\begin{equation*}
	\begin{split}
		\min \;\; &\sum_{i=1}^{N}\sum_{j=1}^{M}c_{ij}x_{ij}\\
		\st&\sum_{j=1}^{M}x_{ij} \leq q_i\quad i=1, \dots, m\\
		& h_j(x) = 0 \quad j=1, \dots, p
	\end{split}
\end{equation*}
%TODO: remark for myself, draw the network
\end{example}
\begin{example}[Training of Neural Networks (Unconstrained Optimization)] Let us start from describing a single (artificial) neuron. In particular, this unit takes as an input a vector and performs a scalar product with the weights of the neuron. In a second step, this value passes through a thresholding function $\sigma$ called activation function. Inspired by real neurons, also this unit propagates its output only if the weighted sum of its inputs (in the natural neuron, the potential difference)  is over a certain threshold. Neural networks stack various neurons in parallel to build a layer and, in turn, concatenate various layer (deep networks). This model is very expressive, in fact already a wide-enough 2-layer neural network is able to approximate any possible function of the input.
	\par Neural networks are the most widely used machine learning model and they can be formally defined as follows. 
	\begin{itemize}
		\item $\mathcal{D}=\{(x_i, y_i)\}_{i=1}^{M}$ be a training set of $M$ instances
		\begin{itemize}
			\item $x_i \in \mathbb{R}^{d}$ $d$ is the amount of features of the dataset (e.g., pixels)
			\item $y_i\in\mathbb{R}$
		\end{itemize}
		\item the $j$-th layer applied on a generic input $n_{j-1}$-dimensional input $x$:\\
		$g_j(x) := \sigma (W_j x +b_j )$, $W_j \in \R^{n_{j-1}\times n_j}$, $b_j \in \R^{n_j}$, $\sigma$ is applied component-wise,\\ e.g., ReLU $\sigma(x)= \begin{cases} 0 & \text{if} \;\;x<0\\ x &\text{otherwise} \end{cases}$ %(ReLU)
		\item the weights: $w = \begin{pmatrix}
			\text{vec}(W_1)\\
			\vdots\\
			\text{vec}(W_L)
		\end{pmatrix}\in \mathbb{R}^{n}$
		\item the network applied on the $i$-th instance: $h_i(w):=g_L \circ \dots \circ g_1(x_i)$ 
		\item the losses: $f_i(w):=\mathcal{L}(h_i(w),y_i)$, e.g., $\mathcal{L}(h_i(w),y_i)= (h_i(w)-y_i)^2$%(sofmax)
	\end{itemize}
	
	\begin{equation*}
		\min_{w\in \mathbb{R}^n} f(w) = \frac{1}{M}\sum_{i=1}^{M} f_i(w)
	\end{equation*}
%TODO: remark for myself, draw the network
\end{example}
\begin{example}[Portfolio Optimization (Convex Quadratic Programming)] Portfolio selection theory studies how to allocate an investor’s available capital into a prefixed set of assets with the aims of maximizing the expected return and minimizing the investment risk. Let $n$ be the number of available assets, let $\mu\in \Rn$ be the vector of expected returns of the assets, and let $Q \in \Rnn$ be a symmetric positive semidefinite matrix whose generic element $q_{ij}$ is the covariance of returns of assets i and j. We assume that the available (normalized) capital is fully invested. Then, let $x \in \Rn$ be the vector of decision variables, where $x_i$ is the fraction of the available capital to be invested into asset $i$, with $i = 1, \dots, n$. 
\par By this notation, $\mu^T x$ is the expected return of the portfolio and $x^TQx$ is the covariance of the portfolio which
can be used as a measure of the risk connected with the investment (diversification of the portfolio). 
In the traditional Markowitz portfolio selection model \cite{markowits52a}, the optimization problem is stated as the
following convex quadratic programming problem
\begin{equation*}
	\begin{split}
		\min \;\; &x^TQx\\
		\st& \mu^Tx\geq \beta \\
		& e^Tx =1\\
		& x\geq 0,
	\end{split}
\end{equation*}
where $\beta$ is the desired expected return of the portfolio and $e\in \Rn$ denotes the column vector of all ones.
\end{example}
\begin{example}[Shortest Path (Discrete Optimization, not covered by this course)] Let $G=(V,E)$ be a graph with $V$ the set of vertices and $E$ the set of edges (directed, $(i,j)\neq(j,i)$, with $i,j$ being two vertices) with $n=|E|$. Let $c_{ij}\geq0$ be the cost of choosing the edge $(i,j)$, e.g., travel time, travel fee, difference in hight, and let $x_{ij}\in \{0,1\}^{n}$ be the variable telling if we select or not the edge $(i,j)$. Also, let $s$ be the starting node and $d$ the destination, then the optimization problem can be stated as
\begin{equation*}
	\begin{split}
		\min \;\; &c^Tx\\
		\st& x\geq 0\\
		& \sum_{j} x_{ij} - \sum_{j} x_{ji} = \begin{cases}1 \quad & i=s\\
			0 & i\neq s,d\\
			-1 & i=d
		\end{cases} \qquad \forall i \in V
	\end{split}
\end{equation*}
%TODO: remark for myself, draw the network
\end{example}

\section{Mathematical Preliminaries}

\subsection{The space $\Rn$}
The vector space $\Rn$  is the set of $n$-dimensional column vectors with real components endowed with the component-wise addition operator
\begin{equation*}
	\begin{pmatrix}
		x_1\\
		x_2\\
		\vdots\\
		x_n
	\end{pmatrix}
+
\begin{pmatrix}
	y_1\\
	y_2\\
	\vdots\\
	y_n
\end{pmatrix}
= 
\begin{pmatrix}
	x_1+y_1\\
	x_2+y_2\\
	\vdots\\
	x_n + y_n
\end{pmatrix}
\end{equation*}
and the scalar product
\begin{equation*}
	\lambda \begin{pmatrix}
		x_1\\
		x_2\\
		\vdots\\
		x_n
	\end{pmatrix}
= 
\begin{pmatrix}
	\lambda x_1\\
	\lambda x_2\\
	\vdots\\
	\lambda x_n
\end{pmatrix}
\end{equation*}
The standard basis of $\Rn$ is $e_1, e_2, \dots, e_n$, where $e_i$ is the $n$-length column vector whose $i$th component is 1 while all the others are 0.\\
The nonnegative orthant $\Rn_+:=\{(x_1,x_2, \dots, x_n)^T: x_1, x_2,\dots,x_n\geq 0\}$.\\
The positive orthant $\Rn_{++}:=\{(x_1,x_2, \dots, x_n)^T: x_1, x_2,\dots,x_n> 0\}$.\\
The set of all real-valued matrices of order $m \times n$ is denoted by $\R^{m\times n}$.\\
For given $x, y \in \Rn$, the closed line segment between $x$ and $y$ is a subset of $\Rn$ denoted by $[x, y]$ and defined as 
\begin{equation*}
	[x,y]:= \{x+a(y-x): \alpha \in[0,1]\}.
\end{equation*}
The open line segment $(x,y)$ is similarly defined as  
\begin{equation*}
	(x,y):= \{x+a(y-x): \alpha \in(0,1)\}.
\end{equation*}
The identity matrix will be denoted by $\Id$, where its dimension will be clear from the context.
\subsection{Inner Products}
\begin{definition}[Inner product] An inner product on $\Rn$ is a map $\inprod{\cdot}{\cdot}: \Rn \times \Rn \to \R$
	with the following properties
	\begin{enumerate}
\item Symmetry: $\inprod{x}{y} = \inprod{y}{x} \quad \forall x, y \in \Rn$;
\item Additivity: $\inprod{x}{y+z} = \inprod{x}{y} + \inprod{x}{z}  \quad \forall x, y,z \in \Rn$;
\item Homogeneity: $\inprod{\lambda x}{y} = \lambda \inprod{x}{y}\quad \forall x, y \in \Rn$;
\item Positive definiteness: $\inprod{x}{x} \geq 0 \quad \forall x \in \Rn$ and $\inprod{x}{x} = 0 \Leftrightarrow x =0.$
	\end{enumerate}
\end{definition}
In this course the inner product we will use is the standard dot product:
\begin{equation*}
	\inprod{x}{y} = x^T y = \sum_{i=1}^{n}x_i y_i \quad \forall x,y\in \Rn
\end{equation*}
Notice that the dot product is not the only inner product in $\Rn$, for instance let $w\in \Rn_{++},$ it is easy to show that the following weighted dot product is also an inner product
\begin{equation*}
\langle x, y\rangle_w = \sum_{i=1}^{n}w_i x_i y_i.
\end{equation*}
\subsection{Vector norms}
\begin{definition}[Vector Norm]
	A norm $||.||$ on $\Rn$ is a function $||.||:\Rn \to \R$ satisfying the following:
	\begin{enumerate}
		\item Nonnegativity: $||x|| \geq 0 \quad \forall x \in \Rn$ and $||x|| = 0 \Leftrightarrow x =0$;
		\item Positive Homogeneity:  $||\lambda x|| = |\lambda| ||x|| \quad \forall x \in \Rn$ and $\lambda\in \R$;
		\item Triangle Inequality: $||x+y|| \leq ||x|| + ||y|| \quad \forall x,y \in \Rn$.
	\end{enumerate}
\end{definition}
One natural way to generate a norm on $\Rn$ is to take any inner product $\inprod{\cdot}{\cdot}$ on $\Rn$ and define the associated norm
\begin{equation*}
	||x||:= \sqrt{\inprod{x}{x}} \quad \forall x \in \Rn
\end{equation*}
which can be proved to be a norm. If the inner product is the dot product, then the associated norm is the Euclidean norm or $l_2$ norm, i.e., $||\cdot||_2$. In the rest of the course, the default norm will be the Euclidean norm and the subscript $2$ will be omitted. This norm belongs to the class of $l_p$ norms (for $p\geq 1$, with $0\leq p<1$ these are quasi-norms) defined by 
\begin{equation*}
	||x||_p:= \sqrt[p]{\sum_{i=1}^{n}|x_i|^p}.
\end{equation*}
By computing the limit of the $l_p$-norm with $p\to\infty$ we achieve the $l_{\infty}$ norm, i.e.,
\begin{equation*}
	||x||_{\infty} := \max_{i\in[n]} |x_i|.
\end{equation*}
where $[n]:=\{1, \dots,n\}$ for any $n \in \N$.
\par An important inequality connecting the dot product of two vectors and their norms is the Cauchy–Schwarz inequality, which will be used frequently throughout the book.
\begin{lemma}[Cauchy–Schwarz inequality]
	For any $x,y \in \Rn$,
	\begin{equation*}
		|x^Ty| \leq ||x||\cdot||y||.
	\end{equation*}
Equality is satisfied if and only if $x$ and $y$ are linearly dependent.
\end{lemma}
\subsection{Matrix norms}
Similarly to vector norms, we can define the concept of a matrix norm.
\begin{definition}[Operator/Matrix Norm]
	A norm $||.||$ on $\Rmn$ is a function $||.||:\Rmn \to \R$ satisfying the following:
	\begin{enumerate}
		\item Nonnegativity: $||A|| \geq 0 \quad \forall A \in \Rmn$ and $||A|| = 0 \Leftrightarrow A =0$;
		\item Positive Homogeneity:  $||\lambda A|| = |\lambda| ||A|| \quad \forall A \in \Rmn$ and $\lambda\in \R$;
		\item Triangle Inequality: $||A+B|| \leq ||A|| + ||B|| \quad \forall A,B \in \Rmn$.
	\end{enumerate}
\end{definition}
Given a matrix $A\in \Rmn$ and two norms $||.||_p$ and $||.||_q$ on $\Rn$ and $\R^m$, respectively, the induced operator/matrix norm $||A||_{a\to b}$ is defined by
\begin{equation*}
	||A||_{p\to q} = \max_{||x||_p=1} ||Ax||_q. 
\end{equation*}
%TODO: remark for myself, point their attention to the concordance of the dimension between A and x
It can be shown that the above definition implies that for any $x \in \Rn$ it holds the inequality
\begin{equation*}
	||Ax||_q \leq ||A||_{p\to q} ||x||_p.
\end{equation*}
An induced matrix norm is indeed a norm in the sense that it satisfies the three properties required above. We refer to the matrix norm $||\cdot||_{p,q}$s the $(p, q)$-norm. When $p=q$, we will simply refer to it as an $p$-norm and omit one of the subscripts in its notation.
\par An important operator norm is the spectral norm, where $p=q=2$. In this case, it can be proved that the operator norm coincide with the maximum singular value of $A \in \Rmn$ (see below for more details on eigenvalues and singular values)
\begin{equation*}
	||A||_2 = ||A||_{2\to2} = \sqrt{\lambda_1 (A^TA)} =: \sigma_1 (A).
\end{equation*}
The default operator norm for this course will be the spectral norm, so when the subscript is omitted we will implicitly refer to this norm.
\par When $p=q=1$, the operator norm reduces to 
\begin{equation*}
	||A||_1 = ||A||_{1\to1} = \max_{j\in[n]} \sum_{i=1}^{m} A_{ij}.
\end{equation*}
This norm is also called maximum absolute column sum norm.
\par When $p=q=\infty$, the operator norm reduces to 
\begin{equation*}
	||A||_\infty = ||A||_{\infty\to\infty} = \max_{j\in[m]} \sum_{i=1}^{n} A_{ij}.
\end{equation*}
This norm is also called the maximum absolute row sum norm.
\par An example of a matrix norm that is not induced by any norm is the Frobenius norm defined by
\begin{equation*}
	||A||_F = \sqrt{\sum_{i=1}^{m} \sum_{j=1}^{n} A^2_{ij}} = ||\text{vec}(A)||,
\end{equation*}
in fact, you can think of this norm as the Euclidean norm of the vector obtained by flattening the matrix $A$ into a vector.

\subsection{Eigenvalues and Eigenvectors}
Let $A \in \Rnn$. Then a nonzero vector $v \in \Rn$ is called an eigenvector of $A$ if there exists a $\lambda \in \Cx$ (the complex field) for which
\begin{equation*}
	A v = \lambda v
\end{equation*}
The scalar $\lambda$ is the eigenvalue corresponding to the eigenvector $v$. In general, real-valued matrices can have complex eigenvalues, but it is well known that all the eigenvalues of symmetric matrices are real. The eigenvalues of a symmetric $n \times n$ matrix $A$ are denoted by
\begin{equation*}
	\lambda_1(A)\geq \lambda_2(A) \geq \dots \geq \lambda_n(A),
\end{equation*}
where $\lambda_1(A)$ is the maximum eigenvalue and $\lambda_n(A)$ is the minimum.
\begin{definition}
	A matrix $U$ is said to be orthogonal when $U^TU =UU^T=\Id$.
\end{definition}
One of the most useful results related to eigenvalues is the spectral decomposition theorem, which states that any symmetric matrix $A$ has an orthonormal basis of eigenvectors.
\begin{theorem}[Spectral Decomposition]
	Let $A\in \Rnn$ be a symmetric matrix and let $v_1,\dots, v_n$ be the eigenvectors corresponding to the eigenvalues $\lambda_1(A), \dots, \lambda_n(A)$. Then $U= \left(v_1 | \dots | v_n\right)$ is an orthonormal matrix and A can be decomposed as follows 
	\begin{equation*}
		A = UDU^T, \quad \with D:=\diag(\lambda_1(A), \dots, \lambda_n(A)).
	\end{equation*}
\end{theorem}
A direct result of the spectral decomposition theorem is that the trace and the determinant of a matrix $A$ can be expressed via its eigenvalues:
\begin{equation*}
	\begin{split}
		\Tr(A) = \sum_{i=1}^n \lambda_i(A)\\
		\det(A) = \prod_{i=1}^n \lambda_i(A)
	\end{split}
\end{equation*}
Another important consequence of the spectral decomposition theorem is the bounding of the so-called Rayleigh quotient. For a symmetric matrix $A \in \Rnn$, the Rayleigh quotient is defined as
\begin{equation*}
	R_A(x) := \frac{x^TAx}{||x||^2} \quad \forall x\neq 0,
\end{equation*}
and its upper and lower bounds are given by the following lemma.
\begin{lemma}
	Let $A\in \Rnn$ be symmetric. Then
	\begin{equation*}
		\lambda_n(A) \leq R_A(x) \leq \lambda_1(A) \quad \forall x\neq 0.
	\end{equation*}
\end{lemma}
%TODO: the proof is a possible exercise
Moreover, we have the following corollary.
\begin{corollary}
	Let $A\in \Rnn$ be symmetric. Then
	\begin{equation*}
		\min_{x\neq0} R_A(x) = \lambda_n(A),
	\end{equation*}
and the eigenvector(s) of $A$ corresponding to the minimal eigenvalue are the minimizers of this problem. In addition,
\begin{equation*}
	\max_{x\neq0} R_A(x) = \lambda_1(A),
\end{equation*}
and the eigenvectors of $A$ corresponding to the maximal eigenvalue are maximizers of this problem.
\end{corollary}
%TODO: the proof is a possible exercise
\subsection{Singular values and Singular Vectors}
Similar objects exist also for non-squared matrices $A \in \Rmn$.
\begin{theorem}[Singular Value Decomposition] Let $A \in \Rmn$. Then there exist orthogonal matrices $U\in \R^{m\times m}$ and $V\in \Rnn$ and the uniquely defined real values
\begin{equation*}
	\sigma_1\geq \sigma_2 \geq \dots \geq \sigma_\ell \geq 0, \quad \with \ell=\min\{m,n\},
\end{equation*}
such that 
\begin{equation*}
	A = U \Sigma V^T \quad \with \Sigma=\diag(\sigma_1, \dots, \sigma_\ell) \in \Rmn.
\end{equation*}
%TODO: notice the dimension of $\Sigma$
\end{theorem}
The matrices $U = (u_1, \dots, u_m)$ and $V = (v_1, \dots, v_m)$ are composed respectively of the left and right singular vectors, thus we can decompose $A$ also as follows
\begin{equation*}
	A = \sum_{j=1}^{\ell} \sigma_j u_jv_j^T.
\end{equation*}
If rank$(A)=r$, then $\sigma_1, \geq \dots, \geq \sigma_r>0$ and $\sigma_{r+1}= \dots=\sigma_\ell=0$ so that the reduced singular value decomposition is
\begin{equation*}
	A = \sum_{j=1}^{r} \sigma_j u_jv_j^T.
\end{equation*}
Also, note that 
\begin{equation*}
	A^T A=(U\Sigma V^T)^T U \Sigma V^T = V \Sigma^T U^T U \Sigma V^T = V \Sigma^T \Sigma V^T = V\diag(\sigma_1^2, \dots, \sigma_\ell^2, 0, \dots, 0) V^T
\end{equation*}
and 
\begin{equation*}
	A A^T=U\Sigma V^T (U \Sigma V^T)^T = U \Sigma V^T V \Sigma^T U^T = U \Sigma \Sigma^T U^T = U\diag(\sigma_1^2, \dots, \sigma_\ell^2, 0, \dots, 0) U^T,
\end{equation*}
%TODO: notice the dimension of the zero padding
which implies that $\sigma_1^2, \dots, \sigma_\ell^2$ are eigenvalues of $A^T A$ and $AA^T$, 
$u_1, \dots, u_m$ are eigenvectors of $AA^T$ and $v_1, \dots, v_n$ are eigenvectors of $A^TA$. In other words the SVD can be obtained from the spectral decomposition of $AA^T$ and $A^TA$. In particular, 
\begin{equation*}
	\sigma_j = \sigma_j(A) = \sqrt{\lambda_j(A^TA)} = \sqrt{\lambda_j(AA^T)}
\end{equation*}
To conclude this section, the following function is called nuclear norm
\begin{equation*}
	||A||_* = \sum_{j=1}^{\ell} \sigma_j(A) \quad \with \ell =\min\{m,n\},
\end{equation*}
and can actually proved to be a norm for $\Rmn$.
%TODO: possible exercise, prove that the nuclear norm is indeed a norm

\subsection{Basic Topological Concepts}
Let us start with the definition of a ball.
\begin{definition}[Open/Closed Ball] The open ball with center $c\in \Rn$ and radius $r \in \R$ is denoted by $B(c,r)$ and defined by 
\begin{equation*}
	B(c,r) := \{x\in\Rn: ||x-c||< r\}
\end{equation*}
The closed ball with center $c\in \Rn$ and radius $r \in \R$ is denoted by $B[c,r]$ and defined by 
\begin{equation*}
B[c,r] := \{x\in\Rn: ||x-c||\leq r\}
\end{equation*}
\end{definition}
The ball $B(c, r)$ for some arbitrary $r>0$ is also referred to as a neighborhood of $c$. The first topological notion we define is that of an interior point of a set. These points have their neighborhood contained in the set.
\begin{definition}[Interior Points] Given a set $S \subseteq\Rn$, a point $c\in S$ is an interior point of $S$ if there exists $r>0$ for which $B(c,r) \subseteq S$.
\end{definition}
The set of all interior points of a given set $S$ is called the interior of the set and is denoted by $\interior(S)$:
\begin{equation*}
	\interior(S) :=\{x\in S: B(x,r)\subseteq S \text{for some } r>0\}.
\end{equation*}
\begin{example}
	Here are some examples 
	\begin{equation*}
		\begin{split}
			\interior(\Rn_+) &= \Rn_{++},\\
			\interior(B(c,r)) &= B(c,r) \quad \with c \in \Rn, r \in \Rn_{++},\\
			\interior([x,y]) &= (x,y)\quad \with x,y \in \Rn, x\neq y.
		\end{split}
	\end{equation*}
\end{example}
\begin{definition}[Open Set]
	An open set is a set that contains only interior points. In other words $S\subseteq \Rn$ is an open set if $\forall x \in S$ there exists $r>0$ such that $B(x,r)\subseteq U$.
\end{definition}
Examples of open sets are $\Rn$, open balls, open line segments, and the positive orthant $\Rn_{++}$. A known result is that a union of any number of open sets is an open set and the intersection of a finite number of open sets is open.
\begin{definition}[Closed Set]
	A set $S\subseteq\Rn$ is said to be closed if it contains all the limits of convergent sequences of points in $S$, that is $S$ is closed if for every sequence of points $\{x_k\} \subseteq S$ satisfying $\displaystyle\lim_{k\to \infty} x_k\!=\! x^*$, it holds that $x^*\in S$.
\end{definition}
Examples of closed sets are the closed balls, closed lines segments and the nonnegative orthant $\R_+$. The space $\Rn$ is both closed and open. An important and useful result states that level sets, as well as contour sets, of
continuous functions are closed.
\begin{proposition}
	Let $f$ be a continuous function over a closed set $S\in\Rn$. Then for any $\alpha \R$, the following sets are closed
	\begin{equation*}
		\begin{split}
			Lev(f, \alpha) :=\{x\in S: f(x)\leq \alpha\}\\
			Con(f, \alpha) :=\{x\in S: f(x)= \alpha\}.
		\end{split}
	\end{equation*}
\end{proposition}
\begin{definition}[Boundary Points]
	Given a set $S\subseteq \Rn$, a boundary point of $S$ is a point $x\Rn$ such that any neighborhood of $x$ contains at least one point in $S$ and at least one point in its complement $S^c$.
\end{definition}
The set of all boundary points of a set $S$ is denoted by $\bd(S)$ and is called the boundary of $S$.
\begin{example}
	Some examples of boundary sets are 
	\begin{equation*}
		\begin{split}
			\bd(B(c,r))&= \bd(B[c,r]) = \{x\Rn: ||x-c||=r\} \quad \with c \in \Rn, r \in \Rn_{++},\\
			\bd(\Rn_{++}) &= \bd(\Rn_+) = \{x\in \Rn_+: \E x_i = 0\},\\
			\bd(\Rn) &= \emptyset.
		\end{split}
	\end{equation*}
\end{example}
\begin{definition}[Closure of a Set]
	The closure of a set $S\in \Rn$ is denoted by $\cl(S)$ and it is the union of $S$ with its boundary
	\begin{equation*}
		\cl(S)=S \cup \bd(S).
	\end{equation*}
Another equivalent definition of $\cl(U)$ is given by the smallest closet set containing $S$, i.e.,
	\begin{equation*}
		\cl(S):= \bigcap \{T: S \subseteq T, T \text{is closed}\}.
	\end{equation*} 
\end{definition}
\begin{example}
	Some examples of closure are 
	\begin{equation*}
		\begin{split}
			\cl(\Rn_{++}) &= \Rn_+,\\
			\cl(B(c,r)) &= B[c,r] \quad \with c \in \Rn, r \in \Rn_{++},\\
			\cl((x,y)) &= [x,y]\quad \with x,y \in \Rn, x\neq y.
		\end{split}
	\end{equation*}
\end{example}
%TODO: is the interior of a open set itself
%TODO: is the closure of a closed set itself
\begin{definition}[Boundedness]
	A set $S$ is bounded when there exists a $M>0$ such that $S\subseteq B(0,M)$.
\end{definition}
\begin{definition}[Compactness]
	A set $S$ is compact when it is bounded and closed.
\end{definition}
Examples of compact sets are closed balls and line segments. The positive orthant is not compact since it is unbounded, and open balls are not compact since they are not closed.
\subsection{Differentiability}
\begin{definition}[Directional Derivative]
	Given a function $f$ defined on a set $S\subseteq \Rn$, a point $x\in \interior(S)$ and a direction $d\in \Rn$, with $d\neq0$, the directional derivative of $f$ at $x$ along $d$ is the below limit
	\begin{equation*}
		\lim_{t\to 0^+} \frac{f(x+td)-f(x)}{t},
	\end{equation*}
when it exists. In this case, the directional derivative is denoted by $f'(x,d)$.
\end{definition}
\begin{definition}[Partial Derivative]
For any $i\in[n]$, the directional derivative of $f$ at $x$ along the direction $e_i$ is called the $i$-th partial derivative and it is denoted by 
\begin{equation*}
	\partiali(x):=\lim_{t\to 0^+} \frac{f(x+te_i)-f(x)}{t}.
\end{equation*}
\end{definition}
\begin{definition}[Gradient]
	If all partial derivatives of a function $f$ exist at a point $x\Rn$, then the gradient of $f$ at $x$ is defined by the column vector consisting of all partial derivatives of $f$:
	\begin{equation*}
		\grad (x) := \begin{pmatrix}
			\frac{\partial f}{\partial x_1}(x)\\
			\frac{\partial f}{\partial x_2}(x)\\
			\vdots\\
			\frac{\partial f}{\partial x_n}(x)
		\end{pmatrix}
	\end{equation*}
\end{definition}
A function $f$ defined on an open set $S \subseteq \Rn$ is called continuously differentiable (or smooth) over $S$, and we write $f\in \C(S)$, if all the partial derivatives exist and are continuous on $S$
%TODO: notice this definition can be extended to nonopen sets
\par An important formula to compute the directional derivative is the following
\begin{equation*}
	f'(x,d) = \grad(x)^Td \quad \forall x \in S, d\in \Rn.
\end{equation*}
The partial derivatives $\partiali$ are themselves real-valued functions that can be partially differentiated. The second order $(i,j)$-th partial derivative of $f$ at $x\in S$ (if exists) is defined by
\begin{equation*}
	\frac{\partial^2 f}{\partial x_i \partial x_j}(x) := \frac{\partial \left(\frac{\partial f}{\partial x_j}\right)}{\partial x_i}.
\end{equation*}
A function $f$ defined on an open set $S \Rn$ is called twice continuously differentiable over $S$, and we write $f\in \Cii(S)$, if all the second order partial derivatives exist and are continuous over $S$. When $f$ is twice continuously differentiable, the second order partial derivatives are symmetric, meaning that for any $i\neq j$ and any $x\in S$
\begin{equation*}
	\frac{\partial^2}{\partial x_i \partial x_j} (x) = \frac{\partial^2}{\partial x_j \partial x_i} (x).
\end{equation*}
\begin{definition}[Hessian]
	Assuming $f\in \Cii(S)$, we can compute the Hessian of $f$ at a point $x\in S$ as the symmetric $n\times n$ matrix
	\begin{equation*}
		\hess (x) = \begin{pmatrix}
			\frac{\partial^2}{\partial x_1^2 } (x) & \dots & \frac{\partial^2}{\partial x_1 x_n } (x)\\
			\vdots & \ddots & \vdots\\
			\frac{\partial^2}{\partial x_n x_1 } (x) & \dots & \frac{\partial^2}{\partial x_n^2 } (x)
		\end{pmatrix}
	.
	\end{equation*}
\end{definition}
In the following we list a few important theorems that are consequence of Taylor's approximation theorem and that might be useful in the rest of the course. 
\begin{theorem}[Linear Approximation]
	Let $f\in \C(S)$, with $S\subseteq \Rn$ open, and let $x\in S, r>0$ satisfy $B(x,r)\subseteq S$. Then for any $y\in B(x,r)$, we have 
	\begin{equation*}
		f(y) = f(x)+ \grad (x)^T(y-x) + o(||y-x||),
	\end{equation*}
where $o(\cdot):\R_+\to \R$ is a one dimensional function satisfying $\frac{o(t)}{t} \to 0$ as $t\to0^+$.
\end{theorem}
\begin{theorem}[Quadratic Approximation]
Let $f\in \Cii(S)$, with $S\subseteq \Rn$ open, and let $x\in S, r>0$ satisfy $B(x,r)\subseteq S$. Then for any $y\in B(x,r)$, we have 
\begin{equation*}
	f(y) = f(x)+ \grad (x)^T(y-x) +\frac{1}{2}(y-x)^T\hess(x)(y-x) +  o(||y-x||^2).
\end{equation*}
\end{theorem}
\begin{theorem}[Mean Value Theorem]
	Let $f\in \C(S)$, with $S\subseteq \Rn$ open, and let $x\in S, r>0$ satisfy $B(x,r)\subseteq S$. Then for any $y\in B(x,r)$, there exists a $c\in [x,y]$ such that 
	\begin{equation*}
		f(y) = f(x)+ \grad (c)^T(y-x).
	\end{equation*}
\end{theorem}
%TODO: draw the one dimensiona MVT
\begin{theorem}
	Let $f\in \Cii(S)$, with $S\subseteq \Rn$ open, and let $x\in S, r>0$ satisfy $B(x,r)\subseteq S$. Then for any $y\in B(x,r)$, there exists a $c\in [x,y]$ such that 
	\begin{equation*}
		f(y) = f(x)+ \grad (x)^T(y-x) +\frac{1}{2}(y-x)^T\hess(c)(y-x).
	\end{equation*}
\end{theorem}
\section{Optimality Condition for Unconstrained Optimization}
\subsection{Global and Local Optima}

Despite our main interest in this section is to discuss minimum and maximum points of a function over the entire space (unconstrained), we here still present the more general definition of global minimum and maximum points of a function over a given set. 
\begin{definition}[Global Minimum and Maximum] 
	Let $f:S\to \R$, then
	\begin{enumerate}
		\item $x^* \in S$ is a global minimum point of $f$ over $S$ if $f(x^*)\leq f(x) \;\forall x \in S$;
		\item $x^* \in S$ is a strict global minimum point of $f$ over $S$ if $f(x^*)< f(x) \;\forall x \in S,\; x\neq x^*$;
		\item $x^* \in S$ is a global maximum point of $f$ over $S$ if $f(x^*)\geq f(x) \;\forall x \in S$;
		\item $x^* \in S$ is a strict global maximum point of $f$ over $S$ if $f(x^*)> f(x) \;\forall x \in S,\; x\neq x^*$.
	\end{enumerate}
\end{definition}
The set $S$ on which the optimization of $f$ is performed is also called the feasible set, and any point $x \in S$ is called feasible solution. It is common to refer to a global minimum point as a minimizer or a global minimizer.
\par Despite they might not be attained, the minimal and maximal value of $f$ over $S$ is the classical notation to refer to the infimum and supremum of $f$ over $S$. These values are always unique, while there might exist many global minima/maxima attaining the same value of $f$, i.e. $f(x^*)$. The sets of all global minimizers/maximizers of $f$ over $S$ are denoted by
\begin{equation*}
	\argmin \{f(x): x\in S\}\qquad \argmax \{f(x): x\in S\}.
\end{equation*}
A vector $x^* \in S$ is called a global optimum of $f$ over $S$ if it is either a global minimum or a global maximum.
%\par Observe that maximization problems can be easily transformed in minimization problems
%$$\min_x f(x) = -\max_x -f(x).$$
%Thus from now on we will focus on the latter. 

Our main task will usually be to find and study global minimum or maximum points; however, most of the theoretical results only characterize local minima and maxima which are optimal points with respect to a neighborhood of the point of interest. 
\begin{definition}[Local Minima and Maxima] 
	Let $f:S\to \R$, then
	\begin{enumerate}
		\item $x^* \in S$ is a local minimum point of $f$ over $S$ if there exists $r>0$ for which \\$f(x^*)\leq f(x) \;\;\forall x \in S\cup B(x^*,r)$;
		\item $x^* \in S$ is a strict local minimum point of $f$ over $S$ if there exists $r>0$ for which \\$f(x^*)< f(x) \;\forall x \in S\cup B(x^*,r),\; x\neq x^*$;
		\item $x^* \in S$ is a local maximum point of $f$ over $S$ if there exists $r>0$ for which \\ $f(x^*)\geq f(x) \;\forall x \in S\cup B(x^*,r)$;
		\item $x^* \in S$ is a strict local maximum point of $f$ over $S$ if there exists $r>0$ for which \\ $f(x^*)> f(x) \;\forall x \in S\cup B(x^*,r),\; x\neq x^*$.
	\end{enumerate}
\end{definition}
Notice that a global minimum (maximum) point is also a local minimum (maximum) point.
\par A well-known result (Fermat's theorem) is that for a one-dimensional function $f$ defined and differentiable
over an interval $(a, b)$, if a point $x^*\in (a, b)$ is a local optimum, then $f'(x^*) = 0$. The multidimensional extension of this result states that the gradient is zero at local optima. We refer to such an optimality condition as a first order optimality condition, as it is expressed in terms of the first order derivatives.
\begin{theorem}[First Order Optimality Condition]\label{thm:first_order_optimality}
	Let $f:S\to \R$ be a function defined on $S\subseteq \Rn$. Suppose that $x^*\in\interior(S)$ is a local optimum and that all partial derivatives of $f$ exist at $x^*$. Then $\grad(x)=0$.
\end{theorem}
\begin{proof}
	Let $i\in [n]$ and consider the one-dimensional function $g(t)=f(x^*+te_i)$. Note that $g$ is differentiable at $t=0$ and by definition $g'(0) = \partiali(x^*)$. Since $x^*$ is a local optimum of $f$, if follows that $t=0$ is a local optimum of $g$, which, by Fermat's Theorem, directly implies that $g'(0) = 0$ and thus also $\partiali(x^*)=0$. Now, since the argument can be repeated for all $i\in[n]$, the result $\nabla f(x^*) = 0$ follows.
\end{proof}
Theorem \ref{thm:first_order_optimality} presents a necessary optimality condition: the gradient vanishes at all local optimum points, which are interior points of the domain of the function. However, the reverse claim is not true as there could be points which are not local optimum points, whose gradient is zero. The points in which the gradient vanishes are called stationary points and they are the only candidates for local optima among the points in the interior of the domain of the function.
\begin{definition}[Stationary Points]
	Let $f:S\to \R$ be a function defined on a set $S\subseteq \Rn$. Suppose that $x^* \interior(S)$ and that $f$ is differentiable in a neighborhood of $x^*$. Then $x^*$ is called stationary point of $f$ if $\grad(x^*)=0$.
\end{definition}
\subsection{Classification of Matrices}
\begin{definition}[Positive Semidefiniteness]
	A symmetric matrix $A\in \Rnn$ is positive semidefinite, denoted by $A \succcurlyeq0$, if $x^TAx\geq 0$ for every $x\in \Rn$.
\end{definition}
\begin{definition}[Positive Definiteness]
	A symmetric matrix $A\in \Rnn$ is positive definite, denoted by $A \succ0$, if $x^TAx> 0$ for every $x\in \Rn, x\neq 0$.
\end{definition}

\noindent In this course, a positive definite or semidefinite matrix is always assumed to be symmetric. The same definitions can be given for negative (semi)definite matrices. Notice that a matrix $A$ is negative (semi)definite if and only if $-A$ is positive (semi)definite. A more complicated matrix is instead the following.
\begin{definition}[Indefiniteness]
	A symmetric matrix $A\in \Rnn$ is indefinite, denoted by $A \succcurlyeq0$, if there exist $x,y \in \Rn$ such that $x^TAx> 0$ and $y^TAy<0$.
\end{definition}
Indefinite matrices are easier to recognize than positive (semi)definite ones. To better characterize the latter, we need to look the eigenvalues of the matrix.
\begin{theorem}[Eigenvalue characterization theorem]
	Let $A\in \Rnn$ be a symmetric matrix. Then
	\begin{enumerate}
		\item $A$ is positive definite if and only if all its eigenvalues are positive,
		\item $A$ is positive semidefinite if and only if all its eigenvalues are nonnegative,
		\item $A$ is indefinite if and only if it has at least one positive eigenvalue and at least one
		negative eigenvalue.
	\end{enumerate}
\end{theorem}










\bibliographystyle{plain}
\bibliography{../biblio}
\end{document}