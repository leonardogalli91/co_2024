\documentclass[10pt,a4paper]{article}
\usepackage[utf8]{inputenc}
\usepackage[T1]{fontenc}
\usepackage{amsmath}
\usepackage{amsthm}
\usepackage{amssymb}
\usepackage{graphicx}
\usepackage{mathtools}
\usepackage[ruled,vlined,linesnumbered]{algorithm2e}
\usepackage[left=2.50cm, right=2.50cm, top=2.0cm, bottom=2.0cm]{geometry}


\makeatother
\DeclareMathOperator*{\argmin}{argmin}
\DeclareMathOperator*{\Max}{\text{max}}
\DeclareMathOperator*{\E}{\mathbb{E}}
\newcommand{\Ei}[1]{\mathbb{E}_{#1}}
\DeclareMathOperator*{\Eik}{\mathbb{E}_{\mathit{i_k}}}
\DeclareMathOperator*{\Eikplus}{\mathbb{E}_{\mathit{i_{k+1}}}}
\newcommand{\Eiplus}[1]{\mathbb{E}_{i_{#1}}}
\DeclareMathOperator*{\LC}{\text{\textup{LC}}^1}
\DeclareMathOperator*{\grad}{\mathit{\nabla \!f}}
\DeclareMathOperator*{\argmax}{arg\;max}
\DeclareMathOperator*{\gradik}{\mathit{\nabla\!\fik}}
\DeclareMathOperator*{\Lmax}{\mathit{L_{max}}}
\newcommand{\R}{\mathbb{R}}
\newcommand{\st}{\text{s.t.} \;\;\;}
\newcommand{\Rn}{\mathbb{R}^n}
\newcommand{\ikplus}{i_{k+1}}
\newcommand{\wrefi}[2]{w_{r(#1, i_{#2})}}
\newcommand{\wref}[1]{\wrefi{#1}{#1}}
\newcommand{\fik}{f_{i_k}}
\newcommand{\fikofwstar}{\fik(w^*)}
\newcommand{\fofwstar}{f(w^*)}
\newcommand{\fii}[1]{f_{i_{#1}}}
\newcommand{\fikplus}{f_{i_{k+1}}}
\newcommand{\fiplus}[1]{f_{i_{#1}}}
\newcommand{\fimax}{f_i^{\text{max}}}
\newcommand{\fikmax}{\fik^{\text{max}}}
\newcommand{\bmax}{b_{\text{max}}}
\newcommand{\fmax}{f^{\text{max}}}
\newcommand{\fmaxk}[1]{f_{i_{#1}}^{\text{max}}}
\newcommand{\Lik}{L_{i_k}}
\newcommand{\Cik}{C_{i_k}}
\newcommand{\deltak}{\delta^{l_k}}
\newcommand{\deltakplus}{\delta^{l_k+1}}
\newcommand{\deltakminus}{\delta^{l_k-1}}
\newcommand{\etatilde}{\tilde{\eta}_{k,0}}
\newcommand{\muik}{\mu_{i_k}}
\newcommand{\etamax}{\eta^{\text{max}}}
\newcommand{\etamaxx}{\bar{\eta}^{\text{max}}}
\newcommand{\etamin}{\eta^{\text{min}}}
\newcommand{\etaminn}{\bar{\eta}^{\text{min}}}
\newcommand{\minimum}[2]{\min \left\{ #1, #2 \right \} }
\newcommand{\maximum}[2]{\max \left\{ #1, #2 \right \} }
\newcommand{\W}[1]{{\scriptscriptstyle W #1}}
\newcommand{\gradi}[1]{\nabla f_{i_{#1}} (w_{i_{#1}}) }
\makeatletter

\newtheorem{assumption}{Assumption}
\newtheorem{lemma}{Lemma}
\newtheorem{proposition}{Proposition}
\newtheorem{theorem}{Theorem}
\newtheorem{corollary}{Corollary}
\newtheorem{remark}{Remark}


\newcommand{\imgS}{.26}
\newcommand{\dir}{exp1/}
\newcommand{\model}{mlp}
\newcommand{\modelname}{mlp}


\newcommand{\mlp}{{\texttt{mnist|mlp}}}
\newcommand{\res}{{\texttt{cifar10|resnet34}}}
\newcommand{\dense}{{\texttt{cifar10|densenet121}}}
\newcommand{\ress}{{\texttt{cifar100|resnet34}}}
\newcommand{\denses}{{\texttt{cifar100|densenet121}}}
\newcommand{\fashion}{{\texttt{fashion|effb1}}}
\newcommand{\svhn}{{\texttt{svhn|wrn}}}
\newcommand{\wiki}{{\texttt{wiki2|encoder}}}
\newcommand{\ptb}{{\texttt{ptb|xl}}}
\newcommand{\mushrooms}{{\texttt{mushrooms}}}
\newcommand{\rcvone}{{\texttt{rcv1}}}
\newcommand{\ijcnn}{{\texttt{ijcnn}}}
\newcommand{\weighta}{{\texttt{w8a}}}

\title{Continuous Optimization}
\author{Chapter 3: Constrained Optimization}
\date{}
\begin{document}
	\maketitle
\section{Definitions}
	In this chapter we will consider constrained optimization problems with the following shape
	\begin{equation}\label{eq:problem}
		\begin{split}
			\min \;\; &f(x)\\
			\st& x \in C
		\end{split}
	\end{equation}
\begin{definition}[Convex Set]
	A set $C$ is said to be convex if given $x_1,x_2\in C$ and $\lambda\in [0,1]$, then $\lambda x_1 +(1-\lambda) x_2 \in C.$
\end{definition}
\begin{definition}[Convex Function]
A function $f:C\to \R$ defined on a convex set $C$ is said to be convex if given $x_1,x_2\in C$ and $\lambda\in [0,1]$, then 
\begin{equation*}
	f(\lambda x_1 +(1-\lambda) x_2) \leq \lambda f(x_1) +(1-\lambda) f(x_2).
\end{equation*}
\end{definition}
\begin{definition}[Strictly Convex Function]
A function $f:C\to \R$ defined on a convex set $C$ is said to be strictly convex if given $x_1,x_2\in C$ and $\lambda\in [0,1]$, then 
\begin{equation*}
	f(\lambda x_1 +(1-\lambda) x_2) < \lambda f(x_1) +(1-\lambda) f(x_2).
\end{equation*}
\end{definition}
\noindent A function is called concave if $-f$ is convex and strictly concave if $-f$ is strictly convex.
\section{First Order Characterizations of Convex Functions}
\begin{theorem}[Gradient inequality for convex functions]\label{eq:gradient_ineq}
	Let $f\in \C(C)$, where $C$ is convex. Then f is convex over $C$ if and only if
	\begin{equation*}
		f(x) +\grad(x)^T(y-x)\leq f(y) \quad \forall x, y\in C.
	\end{equation*}
\end{theorem}
\begin{proof}
	Exercise.
\end{proof}
\begin{proposition}[Sufficiency of stationarity under convexity]
	Let $f\in \C(C)$, where $C\subseteq\Rn$ is convex. Suppose that $\nabla f(x^*)=0$ for some $x^*\in C$. Then $x^*$ is a global minimizer of $f$ over $C$.
\end{proposition}


\section{Stationarity}

\begin{definition}[Stationary points of convex constrained problems]
	Let $f\in \C(C)$, where $C$ is closed and convex. Then $x^*$ is a stationary point of \eqref{eq:problem} if $\grad(x^*)(x-x^*)\geq 0 \; \forall x\in C.$ 
\end{definition}
\noindent In words, this means that there are no feasible descent directions of $f$ at $x^*$. This suggests that stationarity is in fact a necessary condition for a local minimum of \eqref{eq:problem}.
\begin{theorem}[Stationarity as necessary optimality condition]\label{thm:stationarity}
	Let $f\in \C(C)$, where $C$ is closed and convex and let $x^*$ be a local minimum of \eqref{eq:problem}. Then $x^*$ is a stationary point of \eqref{eq:problem}.
\end{theorem}
\begin{proof}
	Let $x^*$ be a local minimum of $f$ and assume by contradiction that is not a stationary point of \eqref{eq:problem}. Then there exists $x\in C$ such that $\grad(x^*)(x-x^*)< 0$. Therefore, $f'(x,d)<0$, where $d=x-x^*$. Hence, by Lemma 1.1 of Chapter 2, there exists $\epsilon\in(0,1)$ such that $f(x^*+td)<f(x^*)\;\forall t\in(0,\epsilon).$ Since $C$ is convex, we have that $x+td = (1-t)x^*+tx\in C$, leading to the conclusion that $x^*$ is not a local optimum of \eqref{eq:problem}, which is a contradiction.
\end{proof}

\begin{theorem}[Stationarity as necessary optimality condition]
	Let $f\in \C(C)$, where $C$ is closed and convex and $f$ is also convex. Let $x^*$ be a local minimum of \eqref{eq:problem}. Then $x^*$ is a stationary point of \eqref{eq:problem} iff $x^*$ is a optimal solution of \eqref{eq:problem}.
\end{theorem}
\begin{proof}
	The necessity of the stationarity condition follows from Theorem \ref{thm:stationarity}. To prove the sufficiency, assume that $x^*$ is a stationary point of \eqref{eq:problem} and let $x\in C$. Then
	\begin{equation*}
		s
	\end{equation*}
\end{proof}

\bibliographystyle{plain}
\bibliography{../biblio}
\end{document}