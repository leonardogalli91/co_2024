\documentclass{ExerciseSheet}

%Set Number of the Exercise sheet and the submission deadline.
\setExerciseSheetNumber{7}
\setSubmissionDate{xx.xx.2024}

%boolean variable to determine whether the solutions should be included
\newif\ifsolutions
\solutionstrue
\solutionsfalse

%We have a figure in this sheet
\usepackage{graphicx}

\begin{document}


%Start with exercises
%-----------------------------------------------------------------------%

%-----------------------------------------------------------------------%
\begin{problem}
The goal of this exercise is to prove Proposition 2.2 of the lecture. \\
That is we consider the function 
\begin{equation*}
f(x)= \frac{1}{2} x^TQx + c^Tx +b,    
\end{equation*}
where $Q$ is positive definite and symmetric and the sequence ${x_k}$ is derived in the following way:
\begin{align*}
    x_{k+1}=x_k-\frac{1}{\mu_k^{BB1}} \nabla f(x_k),
\end{align*}
and $\mu_k^{BB1}:=R_Q(s_k)=\frac{s_k^TQs_k}{\|s_k\|_2^2}$ is the Rayleigh quotient at point $s_k=x_k-x_{k-1}$.
\begin{enumerate}
    \item Prove that for $\lambda_i < \mu_k^{BB1}$ 
    $$\frac{\lambda_i}{\mu_k^{BB1}}-1\geq 1 - \frac{\lambda_1}{\lambda_n}$$ holds.
    \item Let $(\lambda_i, u_i),$ i=1,...,n be a set of eigenvalue and eigenvector pairs (and assume the eigenvectors are orthonormal). \\
    And let $d_i^k$ be the coefficient $$\nabla f(x_k)=\sum_{i=1}^n d_i^k u_i.$$\\
    Find a closed form for each $d_i^k$, in terms of $d_i^1, \lambda_i, \mu_k^{BB1}$
    \item Prove $$\left(d_i^{k+1}\right)^2\leq \delta^2 \left(d_i^k\right)^2,$$
    where $\delta=\frac{\lambda_1}{\lambda_n}-1$
    \item Conclude $$\|\nabla f(x_k)\|_2\leq \delta^k \|\nabla f(x_0)\|_2$$
    \item Follows as the $u_i$ are orthogonal and iteration. 
\end{enumerate}
\end{problem}

\ifsolutions
\vskip 0.3cm

\begin{solution}
\begin{enumerate}
    \item We have $\lambda_n \leq \lambda_i < \mu_k^{BB1}\leq \lambda_1$, this implies
    \begin{align*}
        \frac{\mu_k^{BB1}}{\lambda_i} \leq \frac{\lambda_1}{\lambda_n}
    \end{align*}
    \item We have $\nabla f(x)= Qx + c$, hence $$\nabla f(x_{k+1}) = Q x_{k+1}+ c = Q x_k - Q \frac{1}{\mu_k^{BB1}} \nabla f(x_k) + c = (I - \frac{1}{\mu_k^{BB1}}Q)\nabla f(x_k)$$
    Inductively we get
    \begin{align*}
        \nabla f(x_{k+1})&=\prod_{j=1}^k (I-\frac{1}{\mu_j^{BB1}}Q) \nabla f(x_1) \\
        &=\prod_{j=1}^k (I-\frac{1}{\mu_j^{BB1}}Q)\sum_{i=1}^n d_i^0 u_i \\
        &= \sum_{i=1}^n d_i^1 \prod_{j=1}^k (1-\frac{\lambda_i}{\mu_j^{BB1}})
    \end{align*}
    Therefore $d_i^k = \prod_{j=1}^k (1-\frac{\lambda_i}{\mu_j^{BB1}}) $.
    \item We have the relation 
    $d_i^{k+1}=(1-\frac{\lambda_i}{\mu_k^{BB1}})d_i^k$\\
    As $(1-\frac{\lambda_i}{\mu_k^{BB1}})<\delta$ the claim follows. 
\end{enumerate}
   
\end{solution}

\fi


%-----------------------------------------------------------------------%
\vskip 0.5cm

\begin{exo}
Let $f:\R^n \longrightarrow \R$, $f(x)=\sqrt{1+\|x\|^2}$. Show that $f\in C^{1,1}_1$
\end{exo}

\ifsolutions
\vskip 0.3cm
\begin{solution}
First we calculate the gradient of $f$
\begin{equation*}
    \nabla f(x) = \frac{x}{\sqrt{1+\|x\|^2}}.
\end{equation*}
And now we calculate the Hessian of $f$
\begin{align*}
   \partial x_j \partial x_j f(x) &= \partial x_j \frac{x_i}{\sqrt{1+\|x\|^2}} = \delta_{i,j} \frac{1}{\sqrt{1+\|x\|^2}} - \frac{x_ix_j}{\sqrt{1+\|x\|^2}^3} \\
   \nabla^2 f(x) &= \text{diag}(\frac{1}{\sqrt{1+\|x\|^2}}) - xx^T \frac{1}{\sqrt{1+\|x\|^2}^3}
\end{align*}
Now we calculate the norm of $\nabla^2 f(x)$
\begin{align*}
    \|\nabla^2 f(x)\| &= \sup_{\|z\|=1} z^T\nabla^2 f(x) z \\
    &= (\frac{1}{\sqrt{1+\|x\|^2}}) - z^Txx^Tz \frac{1}{\sqrt{1+\|x\|^2}^3} \\
    &\leq (\frac{1}{\sqrt{1+\|x\|^2}})  \leq 1. 
\end{align*}
\end{solution}

\fi

%-----------------------------------------------------------------------%
\vskip 0.5cm
\begin{exo}
Give an example of a function $f\in C^{1,1}_L$ and a starting point $x_0$ s.t. the problem minimize $f(x)$ has an optimal solution and the gradient with constant step size $t=\frac{2}{L}$ diverges.
\end{exo}


\ifsolutions
\vskip 0.3cm
\begin{solution}
For example $f(x):\R^2 \rightarrow \R, f(x)=0.5(10x_1^2 + x_2^2)$. Then
\begin{equation*}
    \nabla f(x) = \left(
\begin{array}{c}
10x_1\\
x_2\\
\end{array}
\right), \quad L=10.
\end{equation*}
Choose for example $x_0 = (2, 0)$.
\end{solution}
\fi

%-----------------------------------------------------------------------%
\vskip 0.5cm
\begin{exo}
	Let $A\in \R^{n\times n}$, be a positive definite matrix. 
 Denote by $m=\lambda_{\text{min}}(A)$ and $M=\lambda_{\text{max}}(A)$ the smallest and largest eigenvalues of $A$ respectively. 
 \begin{itemize}
     \item What are the eigenvalues of the matrix $A+MmA^{-1}$?
     \item What is the maximum of $\varphi:[m,M]\longrightarrow \R, \varphi(t)=t+\frac{Mm}{t}$
    \item Show for any $\alpha, \beta \in \R$ we have $\alpha \beta \leq \frac{1}{4}(\alpha+\beta)^2$
    \item Show that for any $0\neq x \in \R^n$ the inequality \begin{equation*}
        \frac{(x^Tx)^2}{(x^TAx)(x^TA^{-1}x)} \geq \frac{4 Mm}{(M+m)^2}
    \end{equation*}
    holds.
 \end{itemize}
\end{exo}

\ifsolutions
\vskip 0.3cm
\begin{solution}
\begin{itemize}
    \item $$\lambda_i(A) + \frac{Mm}{\lambda_i(A)}$$ are the eigenvalues of the matrix. 
    \item \begin{align*}
        \varphi'(t) & = 1 - \frac{Mm}{t^2} \\
        \varphi'(t) = 0 & \Leftrightarrow 1 = \frac{Mm}{t^2} \\
        & \Leftrightarrow Mm = t^2 \\
        & \Leftrightarrow \sqrt{Mm} = t. \\
        \varphi''(t) & = 2 \frac{Mm}{t^3}>0,
    \end{align*}
    so this is a minimum. And the maximum is obtained at the end-point of the interval. 
    \begin{equation*}
        \varphi(m) =  m + \frac{Mm}{m} = m + M = \varphi(M).
    \end{equation*}
    \item Instead we show $(\alpha + \beta)^2 \geq 4 \alpha \beta.$ We have
    \begin{equation*}
        0 \leq (\alpha - \beta)^2 = \alpha^2 -2\alpha \beta + \beta^2 = \alpha^2 + \beta^2 + 2 \alpha \beta - 4 \alpha \beta = (\alpha + \beta)^2 - 4 \alpha \beta.
    \end{equation*}
    \item \begin{align*}
        A + MmA^{-1} &\leq (M+m)I \\
        x^TAx + x^TMmA^{-1}x &\leq (M+m)x^Tx\\
        \Rightarrow (x^TAx)(x^TMmA^{-1}x) \leq \frac{1}{4} (x^TAx + x^TMmA^{-1}x)^2 &\leq \frac{1}{4}((M+m)x^Tx)^2 \\
        \Leftrightarrow \frac{4Mm}{(M+m)^2} \leq \frac{(x^Tx)^2}{(x^TAx)(x^TA^{-1}x)}
    \end{align*}
\end{itemize}
\end{solution}
\fi

\end{document}